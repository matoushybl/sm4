% Soubory musí být v kódování, které je nastaveno v příkazu \usepackage[...]{inputenc}

\documentclass[%        Základní nastavení
%  draft,    				  % Testovací překlad
  12pt,       				% Velikost základního písma je 12 bodů
  a4paper,    				% Formát papíru je A4
  oneside,      			% Jednostranný tisk
	%twoside,      			% Dvoustranný tisk (kapitoly a další důležité části tedy začínají na lichých stranách)
	unicode,						% Záložky a metainformace ve výsledném  PDF budou v kódování unicode
]{report}				    	% Dokument třídy 'zpráva', vhodná pro sazbu závěrečných prací s kapitolami

\usepackage[utf8]		  %	Kódování zdrojových souborů je UTF-8
	{inputenc}					% Balíček pro nastavení kódování zdrojových souborů

\usepackage[				% Nastavení geometrie stránky
	bindingoffset=10mm,		% Hřbet pro vazbu
	hmargin={25mm,25mm},	% Vnitřní a vnější okraj
	vmargin={25mm,34mm},	% Horní a dolní okraj
	footskip=17mm,			  % Velikost zápatí
	nohead,					      % Bez záhlaví
	marginparsep=2mm,		  % Vzdálenost marginálií
	marginparwidth=18mm,	% Šířka marginálií
]{geometry}

\usepackage{sectsty}
	%přetypuje nadpisy všech úrovní na bezpatkové, kromě \chapter, která je přenastavena zvlášť v thesis.sty
	\allsectionsfont{\sffamily}

\usepackage{graphicx} % Balíček 'graphicx' pro vkládání obrázků
											% Nutné pro vložení logotypů školy a fakulty

\usepackage[          % Balíček 'acronym' pro sazby zkratek a symbolů
	nohyperlinks				% Nebudou tvořeny hypertextové odkazy do seznamu zkratek
]{acronym}						
											% Nutné pro použití prostředí 'acronym' balíčku 'thesis'

\usepackage[
	hidelinks,
	breaklinks=true,		% Hypertextové odkazy mohou obsahovat zalomení řádku
	hypertexnames=false % Názvy hypertext. odkazů budou tvořeny nezávisle na názvech TeXu
]{hyperref}						% Balíček 'hyperref' pro sazbu hypertextových odkazů
											% Nutné pro použití příkazu 'pdfsettings' balíčku 'thesis'

\usepackage{pdfpages} % Balíček umožňující vkládat stránky z PDF souborů
                      % Nutné při vkládání titulních listů a zadání přímo
                      % ve formátu PDF z informačního systému

\usepackage{enumitem} % Balíček pro nastavení mezerování v odrážkách
  \setlist{topsep=0pt,partopsep=0pt,noitemsep} % konkrétní nastavení

\usepackage{cmap} 		% Balíček cmap zajišťuje, že PDF vytvořené `pdflatexem' je
											% plně "prohledávatelné" a "kopírovatelné"

%\usepackage{upgreek}	% Balíček pro sazbu stojatých řeckých písmem
											%% např. stojaté pí: \uppi
											%% např. stojaté mí: \upmu (použitelné třeba v mikrometrech)
											%% pozor, grafická nekompatibilita s fonty typu Computer Modern!
                      
%\usepackage{amsmath} %balíček pro sabu náročnější matematiky                 

\usepackage{dirtree}	% sazba adresářové struktury
                      % vhodné pro prezentaci obsahu elektronické přílohy (např. CD)
\usepackage{color,soul}
\usepackage{float}
\usepackage[formats]{listings}	% Balíček pro sazbu zdrojových textů
\lstset{              % nastavení
%	Definice jazyka použitého ve výpisech
%    language=[LaTeX]{TeX},	% LaTeX
%	language={Matlab},		% Matlab
	language={C},           % jazyk C
	morekeywords={let,mut,struct,pub,impl,use,assert\_eq,enum,match,if,fn},
	frame=single,
	captionpos=b,
	numbers=left,
	breaklines=true,
	basicstyle=\ttfamily,	% definice základního stylu písma
    tabsize=2,			% definice velikosti tabulátoru
    inputencoding=utf8,         % pro soubory uložené v kódování UTF-8
		columns=fixed,  %fixed nebo flexible,
		fontadjust=true %licovani sloupcu
    extendedchars=true,
	commentstyle=\color{olive},
	keywordstyle=\color{blue},
	stringstyle=\color{red},
    literate=%  definice symbolů s diakritikou
    {á}{{\'a}}1
    {č}{{\v{c}}}1
    {ď}{{\v{d}}}1
    {é}{{\'e}}1
    {ě}{{\v{e}}}1
    {í}{{\'i}}1
    {ň}{{\v{n}}}1
    {ó}{{\'o}}1
    {ř}{{\v{r}}}1
    {š}{{\v{s}}}1
    {ť}{{\v{t}}}1
    {ú}{{\'u}}1
    {ů}{{\r{u}}}1
    {ý}{{\'y}}1
    {ž}{{\v{z}}}1
    {Á}{{\'A}}1
    {Č}{{\v{C}}}1
    {Ď}{{\v{D}}}1
    {É}{{\'E}}1
    {Ě}{{\v{E}}}1
    {Í}{{\'I}}1
    {Ň}{{\v{N}}}1
    {Ó}{{\'O}}1
    {Ř}{{\v{R}}}1
    {Š}{{\v{S}}}1
    {Ť}{{\v{T}}}1
    {Ú}{{\'U}}1
    {Ů}{{\r{U}}}1
    {Ý}{{\'Y}}1
    {Ž}{{\v{Z}}}1
}

\usepackage{epigraph}

%%%%%%%%%%%%%%%%%%%%%%%%%%%%%%%%%%%%%%%%%%%%%%%%%%%%%%%%%%%%%%%%%
%%%%%%      Definice informací o dokumentu             %%%%%%%%%%
%%%%%%%%%%%%%%%%%%%%%%%%%%%%%%%%%%%%%%%%%%%%%%%%%%%%%%%%%%%%%%%%%

% V tomto souboru se nastavují téměř veškeré informace, proměnné mezi studenty:
% jméno, název práce, pohlaví atd.
% Tento soubor je SDÍLENÝ mezi textem práce a prezentací k obhajobě -- netřeba něco nastavovat na dvou místech.

\usepackage[
%%% Z následujících voleb jazyka lze použít pouze jednu
  czech-english,		% originální jazyk je čeština, překlad je anglicky (výchozí)
  %english-czech,	% originální jazyk je angličtina, překlad je česky
  %slovak-english,	% originální jazyk je slovenština, překlad je anglicky
  %english-slovak,	% originální jazyk je angličtina, překlad je slovensky
%
%%% Z následujících voleb typu práce lze použít pouze jednu
  semestral,		  % semestrální práce (nesází se abstrakty, prohlášení, poděkování) (výchozí)
  %bachelor,			%	bakalářská práce
  %master,			  % diplomová práce
  %treatise,			% pojednání o dizertační práci
  %doctoral,			% dizertační práce
%
%%% Z následujících voleb zarovnání objektů lze použít pouze jednu
%  left,				  % rovnice a popisky plovoucích objektů budou zarovnány vlevo
	center,			    % rovnice a popisky plovoucích objektů budou zarovnány na střed (vychozi)
%
]{thesis}   % Balíček pro sazbu studentských prací


%%% Jméno a příjmení autora ve tvaru
%  [tituly před jménem]{Křestní}{Příjmení}[tituly za jménem]
% Pokud osoba nemá titul před/za jménem, smažte celý řetězec '[...]'
\author[Bc.]{Křestní}{Příjmení}

%%% Pohlaví autora/autorky
% (nepoužije se ve variantě english-czech ani english-slovak)
% Číselná hodnota: 1...žena, 0...muž
\gender{0}

%%% Jméno a příjmení vedoucího/školitele včetně titulů
%  [tituly před jménem]{Křestní}{Příjmení}[tituly za jménem]
% Pokud osoba nemá titul před/za jménem, smažte celý řetězec '[...]'
\advisor[prof.\ Ing.]{Křestní}{Příjmení}[CSc.]

%%% Jméno a příjmení oponenta včetně titulů
%  [tituly před jménem]{Křestní}{Příjmení}[tituly za jménem]
% Pokud osoba nemá titul před/za jménem, smažte celý řetězec '[...]'
% Nastavení oponenta se uplatní pouze v prezentaci k obhajobě;
% v případě, že nechcete, aby se na titulním snímku prezentace zobrazoval oponent, pouze příkaz zakomentujte;
% u obhajoby semestrální práce se oponent nezobrazuje (jelikož neexistuje)
\opponent[doc.\ Mgr.]{Křestní}{Příjmení}[Ph.D.]

%%% Název práce
%  Parametr ve složených závorkách {} je název v originálním jazyce,
%  parametr v hranatých závorkách [] je překlad (podle toho jaký je originální jazyk)
\title[Title of Student's Thesis]{Název studentské práce}

%%% Označení oboru studia
%  Parametr ve složených závorkách {} je název oboru v originálním jazyce,
%  parametr v hranatých závorkách [] je překlad
\specialization[Teleinformatics]{Teleinformatika}

%%% Označení ústavu
%  Parametr ve složených závorkách {} je název ústavu v originálním jazyce,
%  parametr v hranatých závorkách [] je překlad
%\department[Department of Control and Instrumentation]{Ústav automatizace a měřicí techniky}
%\department[Department of Biomedical Engineering]{Ústav biomedicínského inženýrství}
%\department[Department of Electrical Power Engineering]{Ústav elektroenergetiky}
%\department[Department of Electrical and Electronic Technology]{Ústav elektrotechnologie}
%\department[Department of Physics]{Ústav fyziky}
%\department[Department of Foreign Languages]{Ústav jazyků}
%\department[Department of Mathematics]{Ústav matematiky}
%\department[Department of Microelectronics]{Ústav mikroelektroniky}
%\department[Department of Radio Electronics]{Ústav radioelektroniky}
%\department[Department of Theoretical and Experimental Electrical Engineering]{Ústav teoretické a experimentální elektrotechniky}
\department[Department of Telecommunications]{Ústav telekomunikací}
%\department[Department of Power Electrical and Electronic Engineering]{Ústav výkonové elektrotechniky a elektroniky}

%%% Označení fakulty
%  Parametr ve složených závorkách {} je název fakulty v originálním jazyce,
%  parametr v hranatých závorkách [] je překlad
%\faculty[Faculty of Architecture]{Fakulta architektury}
\faculty[Faculty of Electrical Engineering and~Communication]{Fakulta elektrotechniky a~komunikačních technologií}
%\faculty[Faculty of Chemistry]{Fakulta chemická}
%\faculty[Faculty of Information Technology]{Fakulta informačních technologií}
%\faculty[Faculty of Business and Management]{Fakulta podnikatelská}
%\faculty[Faculty of Civil Engineering]{Fakulta stavební}
%\faculty[Faculty of Mechanical Engineering]{Fakulta strojního inženýrství}
%\faculty[Faculty of Fine Arts]{Fakulta výtvarných umění}
%
%Nastavení logotypu (v hranatych zavorkach zkracene logo, ve slozenych plne):
\facultylogo[logo/FEKT_zkratka_barevne_PANTONE_CZ]{logo/UTKO_color_PANTONE_CZ}

%%% Rok sepsání práce
\graduateyear{2030}

%%% Datum obhajoby (uplatní se pouze v prezentaci k obhajobě)
\date{11.\,11.\,1980} 

%%% Místo obhajoby
% Na titulních stránkách bude automaticky vysázeno VELKÝMI písmeny (pokud tyto stránky sází šablona)
\city{Brno}

%%% Abstrakt
\abstract[%
Překlad abstraktu
(v~angličtině, pokud je originálním jazykem čeština či slovenština; v~češtině či slovenštině, pokud je originálním jazykem angličtina)
]{%
Abstrakt práce v~originálním jazyce
}

%%% Klíčová slova
\keywrds[%
Překlad klíčových slov
(v~angličtině, pokud je originálním jazykem čeština či slovenština; v~češtině či slovenštině, pokud je originálním jazykem angličtina)
]{%
Klíčová slova v~originálním jazyce
}

%%% Poděkování
\acknowledgement{%
Rád bych poděkoval vedoucímu diplomové práce panu Ing.~XXX YYY, Ph.D.\ za odborné vedení, konzultace, trpělivost a podnětné návrhy k~práci.
}%  % do tohoto souboru doplňte údaje o sobě, druhu práce, názvu...

%%%%%%%%%%%%%%%%%%%%%%%%%%%%%%%%%%%%%%%%%%%%%%%%%%%%%%%%%%%%%%%%%%%%%%%%

%%%%%%%%%%%%%%%%%%%%%%%%%%%%%%%%%%%%%%%%%%%%%%%%%%%%%%%%%%%%%%%%%%%%%%%%
%%%%%%     Nastavení polí ve Vlastnostech dokumentu PDF      %%%%%%%%%%%
%%%%%%%%%%%%%%%%%%%%%%%%%%%%%%%%%%%%%%%%%%%%%%%%%%%%%%%%%%%%%%%%%%%%%%%%
%% Při načteném balíčku 'hyperref' lze použít příkaz '\pdfsettings':
\pdfsettings
%  Nastavení polí je možné provést také ručně příkazem:
%\hypersetup{
%  pdftitle={Název studentské práce},    	% Pole 'Document Title'
%  pdfauthor={Autor studenstké práce},   	% Pole 'Author'
%  pdfsubject={Typ práce}, 						  	% Pole 'Subject'
%  pdfkeywords={Klíčová slova}           	% Pole 'Keywords'
%}
%%%%%%%%%%%%%%%%%%%%%%%%%%%%%%%%%%%%%%%%%%%%%%%%%%%%%%%%%%%%%%%%%%%%%%%

\pdfmapfile{=vafle.map}

%%%%%%%%%%%%%%%%%%%%%%%%%%%%%%%%%%%%%%%%%%%%%%%%%%%%%%%%%%%%%%%%%%%%%%%
%%%%%%%%%%%       Začátek dokumentu               %%%%%%%%%%%%%%%%%%%%%
%%%%%%%%%%%%%%%%%%%%%%%%%%%%%%%%%%%%%%%%%%%%%%%%%%%%%%%%%%%%%%%%%%%%%%%
\begin{document}
\pagestyle{empty} %vypnutí číslování stránek

%% Vložení desek 
\includepdf[pages=1]%  buďto generovaných informačním systémem
  {pdf/desky}% název souboru nesmí obsahovat mezery!
%% NEBO vytvoření desek z balíčku
%\makecover
%%
\oddpage % při dvojstranném tisku přidá prázdnou stránku
% kazdopádně ale:
\setcounter{page}{1} %resetovaní čítače stránek -- desky do číslování nezahrnujeme

%% Vložení titulního listu
\includepdf[pages=1]%    buďto generovaného informačním systémem
  {pdf/title-page}% název souboru nesmí obsahovat mezery!
%% NEBO vytvoření titulní stránky z balíčku
%\maketitle
%%
\oddpage  % při dvojstranném tisku se přidá prázdná stránka
   
%% Vložení zadání
\includepdf[pages=1]%   buďto generovaného informačním systémem
  {pdf/assignment}% název souboru nesmí obsahovat mezery!
%% NEBO lze vytvořit prázdný list příkazem ze šablony
%\patternpage{}%
%	{\sffamily\Huge\centering ZDE VLOŽIT LIST ZADÁNÍ}%
%	{\sffamily\centering Z~důvodu správného číslování stránek}
%%
\oddpage% při dvojstranném tisku se přidá prázdná stránka

%% Vysázení stránky s abstraktem
\makeabstract

\cleardoublepage
\noindent
{\large\sffamily\bfseries\MakeUppercase{Extended Abstract}}
\section*{Úvod}
Tato práce pojednává o vývoji dvoukanálového kontroléru krokových motorů.
V rámci práce je popsán jak vývoj elektroniky, tak vývoj software.
Uvědomujeme si, že kontroléry krokových motorů jsou již mnohokráte vyřešený problém, který má mnoho komerčně dostupných řešení.
Navzdory tomuto faktu jsme se rozhodli takový kontrolér vyvinout, a to ze dvou důvodů - výsledný kontrolér bude používán v rámcí předmětu BPC-PRP, což má jisté nároky na jeho hardware i software, a proto, že jsme se rozhodli naprogramovat firmware a řídicí software v programovacím jazyce Rust.
V současnosti, je většina embedded projektů programována v klasických jazycích - C a C++, s jistými výjimkami v podobě jazyků Python a Ada.
I když jazyky C a C++ jsou vhodné pro embedded vývoj z důvodu snadného přístupu k periferiím, tyto jazyky si s sebou nesou problémy v podobě nedostatečné ochrany před nevalidním přístupem do paměti a velkým množstvím nedefinovaných chování, které jsou mnohdy často závislé na implementaci kompilátoru.
Podle studijí, které uvádíme v originálním úvodu je špatný přístup do paměti zodpovědný až za 70 \% vysoce závažných problémů v prohlížeči Chrome.
Problém s nedefinovaným chováním je sice méně důležitý než špatná práce s pamětí, ale přesto způsobuje problémy zejména v rámci vývoje, kdy prodlužuje jeho čas a tedy i finanční náročnost.

Tyto problémy nejsou ale jen doménou vysokoúrovňových systémů, ale ve velké míře jsou doménou i samotných embedded systémů, kde mohou mít mnohem katastrofálnější následky než v případě oněch vysokoúrovňových systémů.
Věříme, že tyto problémy lze odstranit, nebo alespoň minimalizovat jejich dopady, právě použitím programovacího jazyka Rust, který byl navržen tak, aby předcházel právě chybám při práci pamětí a to i v případě vícevláknových systémů, minimalizoval nedefinovaná chování a to vše aniž by to nějak ovlivnilo výkon programu nebo jeho paměťovou náročnost.

Myslíme si, že pokrokové myšlenky a nástroje, které jazyk nabízí mohou do embedded systémů přinést mnohem více bezpečnosti a spolehlivosti než je nyní možné dosáhnout s konvenčními nástroji.
To platí zejména v současnosti, kdy složitost embedded systémů neustále roste a zvyšují se požadavky na rychlost vývoje.

\section*{Příbuzné projekty}
V rámci příbuzných projektů popisujeme projekty, které souvisí ať už s vývojem embedded systémů v jazyce Rust, tak se zabývají vývojem kontrolérů pr krokové motory, nebo nám byly přímou inspirací.

Vzhledem k tomu, že výsledný kontrolér má být použit v předmětu BPC-PRP, popisujeme kontrolér, který je v rámci tohoto kurzu již používán - tedy kontrolér KM2, který je založený na mikrokontroléru ATMega8 a s nadřazeným systémem komunikuje pomocí sběrnice I\textsuperscript{2}C.
Vzhledem k použití tohoto mikrokontroléru nelze ale použít vyšší frekvenci I\textsuperscript{2}C než 30 kHz, kvůli chybě ve funkcionalitě clock-stretching.

Dále popisujeme kontrolér KM3, který využívá modernější mikrokontrolér STM32F0, ale zatím nebyl do výuky nasazen.
Pro tento kontrolér jsme již dříve vyvinuly firmware v programovacím jazyce Rust, čímž jsme otestovali schopnost jazyka a jeho nástrojů fungovat na low-endovém procesoru s nedostatkem paměti FLASH.
Použitý procesor byl ale největší slabinou kontroléru, protože neumožňoval implementaci pokročilých funkcí.

Jako další projekt, ve kterém jsme vyvinuli firmware v jazyce Rust popisujeme projekt DCMotor - měnič pro DC motory.
V rámci vývoje jsme nahradili původní firmware naprogramovaný v C++, čímž jsme dosáhli lepších vlastností a odstranění zásadních problémů, jako je třeba extrémní hlučnost motoru.
Měniče s firmware naprogramovaným v programovacím jazyce Rust byly v sedmi kusech nasazeny na roboty pro výstavu Robot 2020, kde zdárně plní svou funkcí.

V rámci projektů, které nám byly inspirací zmiňujeme projekt Mechaduino, který integruje kontrolér pro krokový motor přímo na motor, přičemž je schopen zpětnovazebního řízení pomocí integrovaného enkodéru.

Dále zmiňujeme projekt Flott, který implementuje řízení pohybu krokových motorů v programovacím jazyce Rust.

\section*{Metody}
V rámci použitých metod popisujeme krokové motory a jejich řízení.
Vzhledem k tomu, že jsme se rozhodli použít integrované obvody pro řízení krokových motorů od firmy Trinamic, popisujeme rovněž jejich proprietální technologie, které jsou důležité pro správné nastavení integrovaných obvodů i jejich výběr.
Kromě popisu krokových motorů popisujeme také použité sběrnice, a to CAN bus s protokolem CANOpen, I\textsuperscript{2}C a USB.
Následně popisujeme programovací jazyk Rust a jeho důležité koncepty - proměnné a konstanty, princip vlastnictví a tzv. borrow checker, výčtové typy a pattern matching, datové struktury, traits a generika, makra, standarní knihovnu, testování a build systém Cargo.
Na základě informací o programovacím jazyce se přesouváme k popisu toho, jak lze v tomto jazyce vyvíjet pro embedded systémy.
Diskutujeme podporu pro různé rodiny a jádra mikrokontrolérů, organizace vyvíjející nástroje a knihovny pro embedded Rust, přístup k periferiím, abstrakce pomocí HAL, přístup ke globálnímu stavu (který v rámci bezpečnosti považuje Rust za nebezpečný).
Dále popisujeme asynchronní programování v Rustu, které by mohlo zcela změnit způsob jakým je k embedded software přistupováno.
Důležitou součástí embedded Rustu jsou nástroje, které byli vyvinuty pro snazší práci s mikrokontroléry - jsou jimi například generátor kódu pro přístup k periferiím, extrémně rychlé logování, nebo ochrana paměti před přetečením zásobníku.
Jako nedílnou součást moderního vývoje software popisujeme i automatizované testy a Continuous Integration pro embedded systémy.

Po nezbytném teoretickém úvodu se dostáváme k samotnému vývoji kontroléru.
Nejprve zadefinujeme požadavky na zařízení, které plynou se zadání, ale i z předchozích zkušeností a příbuzných projektů.
Tyto požadavky jsou naprosto nezbytné pro kontrolu plnění cílů projektu.

Poté se dostáváme k vývoji hardware kontroléru, kdy nejprve provedeme rozhodnutí týkající se výběru mikrokontroléru a dalších obvodů a následně vyvineme schéma kontroléru, společně s deskou plošných spojů.
Vývoj elektroniky byl proveden v nástroji KiCAD.

Dále následuje popis vývoje firmware kontroléru, nejprve se věnujeme architektuře firmware, na kterou navazujeme popisem kritických komponent kontroléru.
Velkou pozornost věnujeme popisu vytvořených abstrakcí, díky kterým je firmware kontroléru do značné míry univerzální.
Za zmínku jistě stojí abstrakce pro řízení samotných motorů nebo pro enkodéry.

Na závěr je popsán vývoj řídicí aplikace pro náš kontrolér.
Původní cíl byl vytvořit řídící aplikaci s grafickým uživatelským rozhraním a možností konfigurace, ale vzhledem k nedostatku času byla vytvořena pouze jednoduchá aplikace schopná řídit obě osy kontroléru a to jak v rychlostním, tak v polohovém režimu.

\section*{Výsledky}
Výsledkem práce je funkční kontrolér pro krokové motory, který je schopen tyto motory řídit jak v rychlostním, tak v pozičním módu.
Pro řízení je možné použít buď sběrnici CAN, s protokolem CANOpen, nebo sběrnici I\textsuperscript{2}C.
Konfigurace kontroléru je možná přes integrované USB rozhraní.

V rámci výsledků rovněž popisujeme finální stav projetku společně s přehledem plnění požadavků na kontrolér.
Nedílnou součástí výsledků je i popis programátorského rozhraní a datových modelů, pomocí kterých lze kontrolér řídit.

Jako další součást výsledků popisujeme dvě demonstace funkčnosti kontroléru - jednoduchý lineární posuv řízený přes I\textsuperscript{2}C a malého robota s diferenciálním podvozkem řízeného po sběrnici CAN.

\section*{Závěr}
V rámci této práce jsme navrhli, vyrobili a naprogramovali dvoukanálový kontrolér krokových motorů.
Byly vytvořeny dvě verze hardware, lišící se jak zapojením, tak použitými integrovanými obvody pro řízení krokových motorů, tak designem desky plošných spojů.
Druhá verze hardware je sice mnohem pokročilejší než ta první, i přesto jsme v rámci práce vymysleli další způsoby jak tuto verzi hardware dále vylepšit.

Pro kontrolér jsme vyvinuli firmware v programovacím jazyce Rust.
Momentální verze firmware bohužel momentálně podporuje pouze první verzi hardware, ale doprogramování podpory pro druhou verzi by nemělo být příliš náročné.
V rámci programování jsme využili všech možných nástrojů, které nám jazyk poskytuje - zejména v rámci vývoje abstrakcí, kde jsme hojně využívali traits a generiku.
Věříme, že firmware byl naprogramován dostatečně abstraktně na to, aby jej bylo možné jednoduše rozšiřovat a vylepšovat.
Musíme podotknout, že embedded Rust je již dostatečně vyspělý na to, aby v něm šly bezproblémově a efektivně programovat větší či menší embedded projekty.

Pro jednoduchost testování jsme rovněž vytvořili jednoduchý řídicí software schopný řídit kontrolér jak v rychlostním, tak v pozičním režimu.

Projekt kontroléru plánujeme dále vyvíjet a rozšiřovat, přičemž si uvědomujeme, že i když je momentální výsledek použitelný, tak má k dokonalosti daleko.
Plánujeme napřiklad celý firmware kontrolér automatizovaně testovat, vyrobit třetí a snad poslední verzi hardware, a mnohem více.
Přes toto všechno si myslíme, že by kontrolér šel i v tomto stavu nasadit do výuky jako součást předmětu BPC-PRP.

\makecitation
\makedeclaration
\makeacknowledgement
\tableofcontents
\listoffigures
\listoftables
\lstlistoflistings

\cleardoublepage\pagestyle{plain}   % zapnutí číslování stránek

\chapter*{Introduction \& Motivation}
\phantomsection
\addcontentsline{toc}{chapter}{Introduction \& Motivation}
This thesis describes the design and development of a simple dual-channel stepper-motor controller.
We acknowledge, that driving stepper motors is nowadays a solved problem with many solutions that are commercially available.
Given this fact, we needed to differentiate the project from others.
The first difference is that the target of this project is driving stepper motors in the DCI FEEC BUT's (Department of Control and Instrumentation, Faculty of Electrical Engineering and Communications, Brno University of Technology) Robotics and AI group for students' projects and development of our robots, which imposes some requirements on the \acs{pcb} (\acl{pcb}) size and used technologies.
The second, albeit more important difference is that in contrast with classical embedded systems, this stepper motor controller's firmware and service software will be developed in the Rust programming language.
We believe that this difference is the core of the thesis and further distinguishes itself from other theses and projects on embedded development.

In general, the majority of embedded systems nowadays are developed in the C/C++ programming languages~\cite{cohen_tech_nodate, dubois_programming_nodate, noauthor_embedded_2021}.
There are some examples - there are systems developed in Ada and currently the embedded development in Python is starting to take off in hobby projects~\cite{circuitpython_circuitpython_2021}.
While C and C++ are suitable for development of embedded systems, because they allow for direct hardware access, and the programs written in them can be extremely performant, they carry the problem of memory unsafety and undefined behavior.

Memory unsafe code is nowadays cause of many critical software problems, be it security vulnerabilities or safety hazards.
Recent Chrome browser analysis and report shows that around 70 \% of high severity problems are memory safety problems - meaning problems with pointers and staggering half of these are use-after-free problems~\cite{chromium_projects_memory_nodate}.
Similar results show other statistics, namely from the cURL project\cite{stenberg_half_nodate}.
The notorious Heartbleed bug in the OpenSSL was also a problem of the ability of the program to access memory used by other parts of the program, allowing the attacker to steal confidential data from the memory~\cite{synopsys_heartbleed_2020}.

The problem of undefined behaviors and the inability of the commonly used tools to spot them, can be as harmful as memory safety problems, but in general causes problems mostly during development, making the development take longer and therefore more expensive.
An example of this is when programs behave as was not intended, but with seemingly error-less code.

While the problem of memory safety and \acs{ub}s (\acl{ub}) seem to generally be problems of higher level systems and not embedded systems, we believe that these problems apply to embedded systems as well, as these problems can have as devastating effects as the above mentioned security vulnerabilities.
Imagine a robot uncontrollably spinning and destroying its surroundings because some part of a program overwritten its controls by mistake.

We believe that both of these problems can be solved by using the Rust programming language.
While being a relatively novel language for systems development (development started in 2006), the language is designed to be memory safe, even delivering memory safety for state shared between threads.
Its focus on type safety and strong guarantees about performance of systems programming allows the developers to create powerful and performance, yet in many cases zero-cost (memory or performance) abstractions.
This is especially useful as the complexity of all systems is rising, and we believe that to deliver great systems, the human programmers need to be aided by all available tools.
Even though the language primarily targets high level systems, its design allows for it to be used with bare-metal embedded systems, bringing its advantages to these low level systems.

We also believe that the novel approaches brought by the language and its ecosystem could bring improvements to the existing embedded development approaches and also the strictness of the language could bring more safety and reliability to embedded systems.
Some of these approaches can be unit-testing and integration testing, dependency management and embedded-systems-dedicated open-source tooling.

With this information in mind, we decided to develop the controller's firmware and control application in Rust, showcasing the language's advantages and disadvantages.
This project follows the development of firmware for other motor controllers, described in the Chapter~\ref{ch:related_work}, which were presented on the PAIR conference~\cite{faigl_program_nodate}.
Another aim of this project was also to push forward the development of electronics devices at the Robotics and AI research group - using high-performance MCUs (Microcontroller Unit), state-of-the-art stepper drivers and effective 4-layer PCB design and contemporary manufacturing capabilities.

\chapter{Related Work}
\label{ch:related_work}
This chapter describes the current state of the stepper motor controllers in the Robotics \& \acs{ai} group and the efforts to improve it.
Ongoing efforts to develop embedded systems in the Rust programming language are also described, both in the context of our research group and also in general.
Finally, similar projects - either software or hardware-wise are reported.

\section{KM2}
\label{sec:km2}
Nowadays, a second generation of the KM2 stepper motor controller is widely in use in the Robotics and \acs{ai} research group.
It is used primarily by the students of the BPC-PRP course for driving a simple differentially driven robot.
A render of the KM2 controller can be seen in the Figure~\ref{fig:km2render}.
The controller utilizes an ATMega8 paired with two stepper motor controllers DRV8825, that are utilized in the form of breakout boards generally used in the now obsolete 3D printer controlling RAMPS boards.
The motor controller is controlled using the I\textsuperscript{2}C bus.
There are two major shortcomings of the driver - the used \acs{mcu}'s I\textsuperscript{2}C peripheral's clock-stretching is not compatible with Raspberry Pi's, causing problems on clock speeds higher than 30 kHz.
The second shortcoming are the used driver chips which are quite loud and do not support contemporary advanced features.
Overheating is also common with them.

\begin{figure}[H]
    \centering
    \includegraphics[width=0.5\textwidth]{obrazky/km2render}
    \caption{KM2 motor controller render~\cite{burian_km2renderpng_nodate}.}
    \label{fig:km2render}
\end{figure}

\newpage
\section{KM3}
\label{sec:km3}
The KM3 (or KM2-C) was supposed to be a successor to the previously described KM2 controller, and its main goal was to solve the clock stretching problem by utilizing an STM32F031 \acs{mcu}.
Another advantage of this revision was that the breakout boards for motor driver chips were replaced with driver chips soldered directly on the driver \acs{pcb}.
The controller can be seen in the Figure~\ref{fig:km3}.
Even though the new STM32F031 \acs{mcu} was an improvement over the ATMega8, it proved to be the bottleneck for implementing new functionality for the motor controller as the \acs{mcu} has very limited memory, both FLASH and \acs{ram} and also limited peripherals.
An example of these limitations being that the lack of pins made it impossible to directly generate pulses to control the STEP/DIR interface of the motor driver \acs{ic}; therefore the control had to be done manually in the software.
Another problem with this design is that the MCU utilizes a Cortex-M0 core, which means that the support for atomic instructions is missing, making it hard to work with guarantees about memory safety in cases of interrupt routine being called during memory manipulation.

We developed the Rust firmware~\cite{hybl_robotics-butkm3-rs_2020} for this board and concluded that the board and its design might be suitable for the students' robot projects, but it is way too limited to be used in more serious and complex projects.
We also concluded that the hardware and the technology it has been designed upon, as well as its goals, are obsolete, and that we should not pursue the development of this board further.

\begin{figure}[H]
    \centering
    \includegraphics[width=0.5\textwidth]{obrazky/km3}
    \caption{The KM3 motor controller connected to a Raspberry Pi.}
    \label{fig:km3}
\end{figure}

\section{DCMotor}
\label{sec:dcmotor}
The DCMotor is a \acs{dc} (\acl{dc}) motor controller, developed in the Robotics \& \acs{ai} research group.
It was designed for feedback control of \acs{dc} motors, primarily \acs{dc} motors manufactured by Maxon.
The main design goal was to provide a cheaper alternative to the Maxon EPOS motor controllers.
The driver alongside a connected Maxon DC motor can be seen in the Figure~\ref{fig:dcmotor}.
Originally, the firmware for the motors, developed by Ing. František Burian, Ph.D., implemented current control and velocity control.
However, the firmware exhibited unwanted behavior, such as high motor temperature rises and unwanted high-pitch noise.
After consulting the problem with Ing. Lukáš Kopečný Ph.D., we decided to rewrite the firmware in Rust and remove the current controller, with the reasoning that current control of such low inductance motor makes not much sense, and instead, we replaced it with current limiting and failsafe overcurrent motor disabling.
The new firmware, and some hardware modifications were successfully deployed to seven DCMotor drivers as part of the exhibition robots for the Technical Museum in Brno, where they worked better than with the original firmware.

This driver was the first embedded project that used the Rust programming language to develop the firmware.
We believe that using the language was the right choice and made the firmware simpler to use and made it possible to develop it in such short time.
It can be said, that the work on the firmware for this board laid the foundation for the work on this thesis.

\begin{figure}[H]
    \centering
    \includegraphics[width=0.7\textwidth]{obrazky/dcmotor}
    \caption{The DC Motor driver with a connected Maxon DC motor.}
    \label{fig:dcmotor}
\end{figure}

\newpage
\section{Mechaduino}
\label{sec:mechaduino}
Mechaduino is a project that aims to create a feedback-controlled servo motor out of a stepper motor.
The creators achieve that by mounting a PCB on the back of the motor that contains the power stage, a \acs{mcu}, and a 14-bit magnetic encoder~\cite{tropical_labs_mechaduino_2021}.
The mounting on the back of the stepper motor can be seen in the figure~\ref{fig:mechaduino}.
The big advantage that this project brings is the integration of the whole system de-facto into the motor, removing any need for a separate controller board.
On the other hand, the controller doesn't leverage any existing stepper motor controller solution and instead implements the winding control manually.
When compared to our proposed solution, the Mechaduino has many advantages even though it is only capable of controlling only one motor, it contains an encoder for feedback control and implements servo control algorithms out of the box.
On the other hand, our proposed solution leverages a state-of-the-art stepper motor controller \acs{ic}s (\acl{ic}), making it potentially less error-prone and better for future use and development.
If a semestral thesis and preliminary market research was preceding this project, the Mechaduino would provide valuable information to improve the design of our project.

\begin{figure}[H]
    \centering
    \includegraphics[width=0.7\textwidth]{obrazky/mechaduino}
    \caption{The Mechaduino controller boards mounted on the back of a stepper motors~\cite{tropical_labs_mechaduino_2021}.}
    \label{fig:mechaduino}
\end{figure}

\newpage
\section{Flott}
\label{sec:flott}
Flott is a set of libraries suitable for developing motion controllers programmed in the Rust programming language\cite{braun_flott_nodate}.
It is a relatively new project, and as of now, it contains an abstraction layer for stepper motors and acceleration ramp generators.
The project is taking a different approach to controlling the stepper motors we are.
It aims to utilize software pulse generation instead of timers and uses a variable step period in ramp generation.
Even though this is a good approach, we chose not to follow this model and instead implement this asynchronously using the \acs{mcu} peripherals.
On the other hand, the Flott project might be a great source of inspiration for future development, and maybe sometimes the SM4 motor controller might utilize it.

\section{Takeaways from Related Work}
\label{sec:related-work-takeaways}
The past stepper motor development efforts in the Robotics \& \acs{ai} research group showed the current solutions weak spots and advantages, which resulted in the following directions of the development of this project:
\begin{itemize}
    \item The project shall use a powerful, modern, and capable \acs{mcu} to fully support various features of the controller even in the future.
    \item The project shall use the state-of-art motor controller \acs{ic}s.
    \item The project shall use the Rust programming language for firmware and control software development.
\end{itemize}

The \acs{dc} Motor project showed us that writing a fully functional embedded firmware is possible and viable option.

The Mechaduino project serves as a great inspiration for what can be achieved in a servo motor based on a stepper motor.

Flott shows us that more people are trying to achieve building motion controllers in Rust and that we can get inspired from them and share knowledge with them.
We've been in contact with the Flott creator and consulted some ideas with them.


\chapter{Methodology}
\label{ch:methods}

\section{Requirements}
\label{sec:requirements}
In Better Embedded System Software\cite{phillip_koopman_better_2010} the author specifies written requirements as one of the most important parts of documentation for any embedded system.
The author describes requirements as rules specifying everything the system must do, everything the system must not do and constraints the system must meet.
According to the book, there are three types of requirements - functional requirements, non-functional requirements and constraints.
An important property of requirements is that they must be easily verifiable and if possible directly measurable.
Also for future reference, every requirement must have a unique number.

\subsection{Functional Requirements}
\label{subsec:func_req}
Functional requirements describe properties that must be provided by the target system, these are implemented either by the firmware or the hardware.

The requirement for the stepper-motor controller are specified as follows:

\begin{itemize}
    \item \textbf{FR-01} When no command specifying motor speed is received for a period of time (configurable, e.g. 1~s), the controller shall stop both motors.
    \item \textbf{FR-02} When multiple communication interfaces are connected, the system shall prioritize CAN bus, then I2C. USB has the lowest priority.
    \item \textbf{FR-03} The controller shall set motor current based on the ramping state.
    When the motor is still, low current shall be set, when the motor is accelerating, high current shall be set, when motor is moving with constant speed, the current shall be set to some medium values.
    These values shall be configurable.
    \item \textbf{FR-04} All relevant values (currents, timings, limits, etc.) shall be configurable via USB or CANOpen SDO protocol.
    \item \textbf{FR-05} The controller shall be able to ramp the speeds using at least trapezoidal ramps, and their parameters shall be configurable.
    \item \textbf{FR-06} The controller shall support control in speed mode as well as position mode.
    \item \textbf{FR-07} The controller shall provide basic electrical safety features - such as fuses and reverse voltage protection.
\end{itemize}

\subsection{Non-functional Requirements}
\label{subsec:nonfunc_req}
Non-functional requirements are properties that the system must have, but are not directly features or functions, but rather properties of the system as a whole.

\begin{itemize}
    \item \textbf{NFR-01} The controller shall provide developer-friendly protocol and data formats.
    \item \textbf{NFR-02} The controller shall be programmable without programmer, ideally using DFU.
    \item \textbf{NFR-03} The controller shall be configurable using a program for personal computers.
    \item \textbf{NFR-04} The firmware shall be easily extensible.
    \item \textbf{NFR-05} The firmware shall employ unit testing for QA.
    \item \textbf{NFR-06} The firmware should utilize hardware in the loop integration testing for QA.
    \item \textbf{NFR-06} The firmware shall be properly documented.
    \item \textbf{NFR-08} The functionality of the controller shall be demonstrated using two distinctive applications.
\end{itemize}

\subsection{Constraints}
\label{subsec:constraints}
Constraints specify limitations on how the system must be built, they specify for example hardware limitations, technologies, protocols, conformance to standards, etc.

\begin{itemize}
    \item \textbf{C-01} The controller shall utilize stepper drivers with silent operation, such as Trinamic Stealth Chop.
    \item \textbf{C-02} The controller shall be able to drive motors with current of up to 2~A.
    \item \textbf{C-03} The controller shall feature CAN bus, I2C bus and USB.
    \item \textbf{C-04} The controller shall utilize two stepper motor drivers.
    \item \textbf{C-05} The controller shall adhere to the CANOpen protocol.
    \item \textbf{C-06} The controller hardware shall be small, ideally smaller than the Raspberry Pi SBC.
    \item \textbf{C-07} The firmware for the controller shall be developed using the Rust programming language.
\end{itemize}

\section{Rust programming language}

\section{Embedded Rust}

\section{Hardware}

\section{Software}

\section{Communication protocols}
\subsection{CANOpen}
\subsection{I2C}
\subsection{USB}
The USB communication with the controller is implemented using the virtual COM port protocol.
Data frames have the following format:

\section{Requirements}
\label{sec:requirements}
In the book Better Embedded System Software\cite{koopman_better_2010}, the author specifies written requirements as one of the most important parts of documentation for any embedded system.
The author describes requirements as rules specifying everything the system must do, everything the system must not do, and constraints the system must meet.
According to the book, there are three types of requirements - functional requirements, non-functional requirements, and constraints.
An essential property of requirements is that they must be easily verifiable and, if possible, directly measurable.
Also, for future reference, every requirement must have a unique number.

\subsection{Functional Requirements}
\label{subsec:func_req}
Functional requirements describe properties that must be provided by the target system.
These are implemented either by the firmware or the hardware.

The requirements for the stepper-motor controller are specified as follows:

\begin{itemize}
    \item \textbf{FR-01} When no command specifying motor speed is received for a period of time (configurable, e.g., 1~s), the controller shall stop both motors.
    \item \textbf{FR-02} When multiple communication interfaces are connected, the system shall prioritize CAN bus, then I2C. USB has the lowest priority.
    \item \textbf{FR-03} The controller shall set motor current based on the ramping state.
    When the motor is still, low current shall be set; when the motor is accelerating, high current shall be set; when the motor is moving with constant speed, the current shall be set to some medium values.
    These values shall be configurable.
    \item \textbf{FR-04} All relevant values (currents, timings, limits, etc.) shall be configurable via USB or CANOpen SDO protocol.
    \item \textbf{FR-05} The controller shall be able to ramp the speeds using at least trapezoidal ramps, and their parameters shall be configurable.
    \item \textbf{FR-06} The controller shall support control in speed mode as well as position mode.
    \item \textbf{FR-07} The controller shall provide basic electrical safety features - such as fuses and reverse voltage protection.
    \item \textbf{FR-08} The controller shall have an interface allowing to connect external encoders (absolute or incremental) during future development.
\end{itemize}

\subsection{Non-functional Requirements}
\label{subsec:nonfunc_req}
Non-functional requirements are properties that the system must have but are not directly features or functions but rather properties of the system as a whole.

\begin{itemize}
    \item \textbf{NFR-01} The controller shall provide developer-friendly protocol and data formats.
    \item \textbf{NFR-02} The controller shall be programmable without a programmer, ideally using \acs{dfu} (\acl{dfu}).
    \item \textbf{NFR-03} The controller shall be configurable using a program for personal computers.
    \item \textbf{NFR-04} The firmware shall be easily extensible.
    \item \textbf{NFR-05} The firmware shall employ unit testing for \acs{qa} (\acl{qa}).
    \item \textbf{NFR-06} The firmware should utilize software in the loop integration testing for \acs{qa}.
    \item \textbf{NFR-07} The firmware shall be properly documented.
    \item \textbf{NFR-08} The functionality of the controller shall be demonstrated using two distinctive applications.
\end{itemize}

\subsection{Constraints}
\label{subsec:constraints}
Constraints specify limitations on how the system must be built.
They specify, for example, hardware limitations, technologies, protocols, conformance to standards, etc.

\begin{itemize}
    \item \textbf{C-01} The controller shall utilize stepper drivers with silent operation, such as Trinamic StealthChop\texttrademark.
    \item \textbf{C-02} The controller shall be able to drive motors with phase current of up to 2~A RMS.
    \item \textbf{C-03} The controller shall feature CAN bus, I2C bus and USB.
    \item \textbf{C-04} The controller shall utilize two stepper motor drivers.
    \item \textbf{C-05} The controller shall adhere to the CANOpen protocol.
    \item \textbf{C-06} The controller hardware shall be small, ideally smaller than the Raspberry Pi SBC.
    \item \textbf{C-07} The firmware for the controller shall be developed using the Rust programming language.
\end{itemize}

\section{Stepper motors}
\label{sec:steppers}
This section gives a brief introduction into stepper motors and their control.
First, stepper motors and their types are described.
Then, a comparison of stepper driver ICs is given and some of the motion control technologies by Trinamic are described.

Stepper motors are a type of DC motors which move in discrete steps\cite{bill_earl_all_nodate,carmine_fiore_stepper_2021}.
Such movement is achieved by their construction - they consist of a stator and a rotor, where the stator is made of coils
(two coils form a phase) wound on ridges, whereas the rotor consists is a ferromagnetic structure - either a permanent
magnet or a variable reluctance iron core\cite{carmine_fiore_stepper_2021}.

\subsection{Working principle}
\label{subsec:stepper_working_principle}
The basic working principle of stepper motors can be seen in the~Figure~\ref{fig:stepper_working_principle}.
In the Figure, we can see a three-phase bidirectional stepper motor.
First, the coils of the stator winding \textbf{A} are energized, which causes the ferromagnetic rotor to align with the magnetic field induced by the phase winding.
In the second step, the winding \textbf{B} is energized, causing the rotor magnetic field to realign with the newly induced magnetic field of the second winding.
This causes the motor to move.
In the next step, the winding \textbf{C} is energized, which again causes realignment of the rotor.
In the following steps, the coils are energized again but with different polarity making the rotor make a full turn.

\begin{figure}[H]
    \centering
    \includegraphics[width=\textwidth]{obrazky/stepper_principle}
    \caption{Working principle of a stepper motor~\cite{carmine_fiore_stepper_2021}.}
    \label{fig:stepper_working_principle}
\end{figure}

\subsection{Rotor}
\label{subsec:rotor}
There are three different constructions of rotors~\cite{carmine_fiore_stepper_2021}:
\begin{itemize}
    \item \textbf{Permanent magnet rotor} - utilizes a permanent magnet in the place of the stator.
    An advantage of this type of stator is good torque and also detent torque (the resistance of the motor shaft when no windings are energized)~\cite{carmine_fiore_stepper_2021}.
    \item \textbf{Variable reluctance rotor} - the rotor consists of a shaped iron core.
    The torques are generally lower and there is no detent torque~\cite{carmine_fiore_stepper_2021}.
    \item \textbf{Hybrid motor} - is created by combining a permanent magnet rotor with a variable reluctance rotor.
    There are two magnetic caps with teeths on top of each other that have an angular shift between them.
    The rotor is magnetized axially~\cite{carmine_fiore_stepper_2021}.
\end{itemize}

\subsection{Stator}
\label{subsec:stator}
The construction of stator depends on the number of phases the motor has.
Every phase consists of two windings and the windings can be center-tapped or not, which determines if the motor is bipolar or unipolar.
With unipolar windings, the center tapped lead is connected to the input voltage and the direction of the magnetic field is controlled by connecting ground to the other leads.
Bipolar motors do not have center tapped lead and the coil itself is controlled using a H-bridge.

\subsection{Phase Winding Energizing Techniques}
\label{subsec:winding_tech}
The way of energizing windings described in the Subsection~\ref{subsec:stepper_working_principle} is only one of four ways of controlling the windings.
This technique, where only one of the phases is energized at a time is called the wave mode.
it was described in detail in the Subsection~\ref{subsec:stepper_working_principle} and the sequence of energizing windings can be seen in the Figure~\ref{fig:stepper_wave_mode}.

\begin{figure}[H]
    \centering
    \includegraphics[width=\textwidth]{obrazky/wave_principle}
    \caption{Controlling stepper motor phase windings in wave mode~\cite{carmine_fiore_stepper_2021}.}
    \label{fig:stepper_wave_mode}
\end{figure}

\newpage
Another way of driving the motor is called the full-step mode.
In this mode, two phase windings are energized at the same time.
Changing the current direction in the winding causes the rotor to realign.
The advantage of this mode is higher torque as the magnetic field is stronger when the two of the phase windings are energized.
The working principle can be seen graphically in the Figure~\ref{fig:stepper_full_step_mode}.

\begin{figure}[H]
    \centering
    \includegraphics[width=\textwidth]{obrazky/full_step_principle}
    \caption{Controlling stepper motor phase windings in full step mode~\cite{carmine_fiore_stepper_2021}.}
    \label{fig:stepper_full_step_mode}
\end{figure}

Combining the wave mode and the full step mode results in a half-step mode.
In contrast to the previous driving modes, the step size of this mode is the half of the previous mode - in case of this virtual motor with permanent magnet motor 90\textdegree, therefore 45\textdegree.
This mode alternates between energizing only one phase winding and energizing both phase windings.
The disadvantage of this mode is that the produced torque is not constant as the torque is different when both phase windings are energized and when only one of them is.
The working principle can be seen in the Figure~\ref{fig:stepper_half_step_mode}.

\begin{figure}[H]
    \centering
    \includegraphics[width=\textwidth]{obrazky/half_step_principle}
    \caption{Controlling stepper motor phase windings in half step mode~\cite{carmine_fiore_stepper_2021}.}
    \label{fig:stepper_half_step_mode}
\end{figure}

The last technique for driving stepper motors is microstepping.
The advantage of this mode is that it reduces step size and has constant torque output~\cite{carmine_fiore_stepper_2021}.
The working principle for this mode is that the current flowing through the phase winding is controlled in some ratio, finely positioning the rotor as can be seen in the Figure~\ref{fig:microstepping}.
Microstepping is nowadays the prevalent way of stepper motor control as it allows for precision control and allows for constant torque.

\begin{figure}[H]
    \centering
    \includegraphics[width=\textwidth]{obrazky/microstepping}
    \caption{Controlling stepper motor phase windings in microstepping mode~\cite{carmine_fiore_stepper_2021}.}
    \label{fig:microstepping}
\end{figure}

\subsection{Stepper motor driver IC comparison}
\label{subsec:stepper_ic}
In order to select a proper driver IC for the SM4 stepper motor controller a simplified comparison was performed.
Since the beginning we were aiming at using the Trinamic made stepper motor driver ICs since there are very good references from the 3D printing community\cite{josef_prusa_original_2017,josef_prusa_original_2019,3daddict_stepper_2020}  for them and we wanted to secure silent function of the motor controller.
With the prior experience with DRV8825 and the A4988 modules described in the Chapter on Related work~\ref{ch:related_work}, we decided to select two drivers from Trinamic - the first hardware revision utilizes the TMC2100 as we aimed for as simple chip as possible, while the second hardware revision utilizes the TMC2226, which is a more modern, and fully featured stepper motor driver, which also has higher peak current.
By selecting these two stepper driver ICs, we are conforming to the requirements \textbf{C-01} and \textbf{C-02}.
The comparison can be seen in the following table:

\begin{sidewaystable}
    \centering
    \begin{tabular}{ |p{2.5cm}|p{2cm}|p{2cm}|p{2cm}|p{2.5cm}|p{2.5cm}|p{2cm}|p{4cm}| }
        \hline
        Name & Operating Voltage & Maximum current & Control Interface & Configuration Interface & Microstepping & Package & Advanced Features \\
        \hline
        \hline
        DRV8825\cite{texas_instruments_drv8825_2014} & 8.2-45~V & max 2.5~A (properly cooled, at 24~V, 25~\textdegree) & STEP/DIR & GPIO & up to 32 & HTSSOP28 & None \\
        \hline
        A4988\cite{allegro_microsystems_a4988_2014} & max 35~V & max 2 A & STEP/DIR & GPIO & up to 16 & QFN28ET & None \\
        \hline
        TMC2100\cite{trinamic_tmc2100-datasheet_2018} & max 46~V & max 2 A (2.5 A peaks, properly cooled) & STEP/DIR & GPIO & up to 256 & TQFP48 / QFN36 & MicroPlyer\texttrademark, SpreadCycle\texttrademark, StealthChop\texttrademark \\
        \hline
        TMC2130\cite{trinamic_tmc2130-datasheet_2018} & max 46~V & max 2 A (2.5 A peaks, properly cooled) & STEP/DIR & SPI & up to 256 & TQFP48 / QFN36 & MicroPlyer, SpreadCycle\texttrademark, StealthChop\texttrademark, ChopSync\texttrademark, CoolStep\texttrademark, StallGuard\texttrademark \\
        \hline
        TMC2209\cite{trinamic_tmc2209_2019} & max 29~V & max 2 A (2.8 A peaks) & STEP/DIR & UART & up to 256 & QFN28 & MicroPlyer\texttrademark, SpreadCycle\texttrademark, StealthChop2\texttrademark, CoolStep\texttrademark, StallGuard4\texttrademark \\
        \hline
        TMC2226\cite{trinamic_tmc2226_2020} & max 29~V & max 2 A (2.8 A peaks) & STEP/DIR & UART & up to 256 & HTSSOP28 & MicroPlyer\texttrademark, SpreadCycle, StealthChop2\texttrademark, CoolStep\texttrademark, StallGuard4\texttrademark \\
        \hline
    \end{tabular}
    \caption{Comparison of stepper motor driver ICs}
    \label{tab:driver_ic_comparison}
\end{sidewaystable}
\newpage
\subsection{Trinamic motion control technologies}
\label{subsec:trinamic_tech}
Since we decided to utilize the Trinamic driver ICs and their stepper control technologies, it is vital to describe these technologies as they have immediate impact on the driver performance and properties.

\subsubsection{MicroPlyer\texttrademark}
MicroPlyer\texttrademark is a microstepping interpolator.
The reason for the interpolator is that the drivers feature 256 microsteps per step and generating the stepping signal would be impractical if not impossible for some systems.
The driver is configured with the number of microsteps that the driver will consider a full step and the MicroPlyer\texttrademark interpolates the rest of the microsteps up to the 256 microsteps per step\cite{trinamic_microstepping_nodate}.

\subsubsection{Voltage Chopper Modes - SpreadCycle\texttrademark, StealthChop\texttrademark}
In order to define chopper modes, we first need to define the current control modes for a bipolar stepper motor.
The current control modes are the ON-phase, fast decay and slow decay.
These modes can be seen in the Figure~\ref{fig:current_phases}.

\begin{figure}[H]
    \centering
    \includegraphics[width=\textwidth]{obrazky/winding_modes}
    \caption{Stepper motor winding control modes~\cite{trinamic_chopper_nodate}.}
    \label{fig:current_phases}
\end{figure}

The current in the winding is controlled using voltage choppers.
First, a very high voltage is applied to the winding, which causes current rise in the winding, when the current exceeds specific limit, the voltage is chopped (turned off).
When the current drops below a specified limit, the very high voltage is turned back on.
Using this approach, it is possible to maintain a relatively constant current in the winding\cite{trinamic_chopper_nodate}.

When using a Constant T\_OFF Current Chopper, the basic chopper principle is enhanced by first, energizing the winding, then utilizing fast-decay and then slow decay.
This is a commonly used chopper mode as it is quite simple, but it causes motor vibration, and the high pitch noise.
This problem is caused by the relationship between the fast decay and slow decay phase, resulting in the average current being lower than the desired target.
This means that there are moments when the motor has no torque, which in turn causes vibrations\cite{trinamic_chopper_nodate}.
The graph showing the winding current in time can be seen in the Figure~\ref{fig:t_off_chopper}.

\begin{figure}[H]
    \centering
    \includegraphics[width=\textwidth]{obrazky/t_off_chopper}
    \caption{Constant T\_OFF Chopper Mode~\cite{trinamic_chopper_nodate}.}
    \label{fig:t_off_chopper}
\end{figure}

The SpreadCycle\texttrademark current chopper is an improvement over the Constant T\_OFF Current chopper.
According to Trinamic, it automatically applies a fitting relation between slow decay and fast decay to create the optimal fast decay for that cycle\cite{trinamic_chopper_nodate}.
This technique leads to the average current matching the target current and the current wave resembles sine wave.
This technology also remains effective at higher RPMs, where the classic constant T\_OFF Chopper shows current deformations\cite{trinamic_chopper_nodate}.
The current in time can be seen in the Figure~\ref{fig:spread_cycle}.

\begin{figure}[H]
    \centering
    \includegraphics[width=\textwidth]{obrazky/spread_cycle}
    \caption{The SpreadCycle\texttrademark Mode current graph~\cite{trinamic_chopper_nodate}.}
    \label{fig:spread_cycle}
\end{figure}

StealthChop\texttrademark is the most advanced voltage chopper technology Trinamic drivers provide.
The chopper completely silences stepper motors by eliminating the noise cause by unsynchronized motor coil chopper operation, PWM jitter and regulation noise at the sense resistors\cite{trinamic_chopper_nodate}.
The chopper modulates the current using the PWM duty cycle, which minimizes the current ripple\cite{trinamic_chopper_nodate}.
Adjusting the PWM duty cycle also results in a perfect current sine wave and minimizing the current ripple minimizes Eddy currents in the stator, which in turn leads to less power loss and increase efficiency\cite{trinamic_chopper_nodate}.


\subsubsection{StallGuard\texttrademark}
The StallGuard\texttrademark technology utilizes the back EMF (ElectroMotive Force) to analyze the load of the motor.
This provides the drivers with a sensorless load measurement.
This technology may be utilized for sensorless homing, self-calibration or distance measurement.
The StallGuard\texttrademark technology also prevents step loss when the axis is obstructed\cite{trinamic_trinamic_nodate}.

\subsubsection{CoolStep\texttrademark}
CoolStep\texttrademark is a technology that adjusts the motor current based on the feedback provided by the StallGuard\texttrademark technology.
This technology always drives the motor at the minimum required current sufficient for driving the actual load.
That leads to reduced current consumption and also reduces heat generation.
The technology also allows for temporary current boosts.
An example of the dependency on the motor current on the load torque can be seen in the Figure~\ref{fig:torque_current}.

\begin{figure}[H]
    \centering
    \includegraphics[width=\textwidth]{obrazky/Trinamic_CoolStep}
    \caption{The Trinamic CoolStep\texttrademark technology~\cite{trinamic_trinamic_nodate}.}
    \label{fig:torque_current}
\end{figure}

\section{Communication protocols}
\label{sec:comm_protocols}
According to the thesis instruction and the requirements \textbf{FR-04}, \textbf{NFR-02}, \textbf{C-03}, \textbf{C-05},  the stepper driver should feature CANOpen and I2C interfaces for control and configuration and the USB interface for configuration.
This section aims to give a brief overview into these communication interfaces and protocols utilized with them.

\subsection{CANOpen}
\label{subsec:canopen}
In this subsection, first the CAN bus physical and data link layer is described.
Secondly, the CANOpen protocol is described alongside with what was implemented for the SM4 motor controller.

% TODO cite
\subsubsection{CAN bus}
The Controlled Area Network is a bus most commonly used in automotive for connecting ECUs (Electronic Control Units) together.
The bus was developed by Bosch and codified into the ISO11898-1 standard\cite{}.
CAN bus utilizes a single differential pair making simplifying the wiring of a complex system consisting of many ECUs.

On the physical layer, the bus has two states - recessive and dominant, where recessive means that the differential voltage between the CANH and CANL signals is less than a minimum threshold voltage, whereas dominant state means that the differential voltage is higher than the minimal threshold voltage.
The dominant state is achieved by sending a logical 0 through the network, while the recessive state is achieved by sending logical 1.
CAN bus utilizes the CSMA/CD media access control protocol, which allows for collision detection and potential retransmission of CAN frames.
For collision detection, it is vital, that the dominant state overrides a recessive one.

There are two types of frames transmitted on the bus - standard frames and extended frames.
These frames differ in the identifier length, where the extended frame allows for 29 bit long identifier in contrast to the standard frame, that allows for only 11 bits.
Identifier length is selected on per-frame basis using the IDE bit in the frame.
Each CAN frame may contain up to 8 bytes of data and the data length is controlled by the four DLC bits in the frame.
The structure of a CAN frame can be seen in the Figure~\ref{fig:can_frame}.

\begin{figure}[H]
    \centering
    \includegraphics[width=\textwidth]{obrazky/can_frame}
    \caption{CAN bus frame with standard identifier~\cite{piembsystech}.}
    \label{fig:can_frame}
\end{figure}

An important bit for CANOpen is the RTR bit, which stands for Remote Transmission Request, when this bit is recessive, there are no data contained in the frame and the frame asks the remote device for data.
As can be seen in the figure, the identifier and the RTR field are part of an Arbitration field, these bytes are used in the shared medium collision detection and control and thanks to this field, frames with lower id have higher priority in the transmission.

\subsection{Object Dictionary}
\label{subsec:object_dictionary}
In a CANOpen device, the Object Dictionary contains the global shared state of a device.
This means that the software responsible for communicating over CANOpen protocols sends data available in the Object Dictionary and writes to it the data it receives.
On the other side, the Object Dictionary serves as a data source for the algorithms and systems running on the device itself.
There are two numbers used to access the values in the Object Dictionary - first the Index - a 16 bit unsigned value, and the SubIndex - an 8 bit unsigned value.
Some of the Index ranges are reserved by the CANOpen specification for predefined parameters such as communication settings, while other Index ranges contain application specific parameters\cite{cia}.

\subsection{CANOpen protocols}
\label{subsec:canopen_proto}
CANOpen is a set of higher level protocols based on the CAN bus physical and link layer.
These provide standardized communication objects (COBs) with specific identifiers (IDs) for time critical processes, communication and network management\cite{cia}.
The most important parts CANOpen protocol are the SYNC protocol, the PDO protocol, the SDO protocol and the NMT protocol.
Another protocols are the EMCY protocol, TimeStamp protocol and LSS protocol.
Every device in a CANOpen network is assigned with a unique ID.
Within the CANOpen protocols, the CAN frames sent to device either target specific device or all of them.
The frames that target specific devices contain the identifier of the frame (e.g.PDO CAN ID) bit-ored with the device ID.

\subsubsection{SYNC protocol}
The SYNC protocol is responsible for synchronizing the communication on the bus.
It initiates the transfer by sending a CAN frame with the identifier \textbf{0x80}, after which every device on the bus sends/receives synchronous data objects, such as PDOs (Process Data Units).
The CAN frame can also contain a single byte containing SYNC number, that can be utilized to conditionally send synchronous data or for more granular synchronization\cite{cia}.
SYNC message is generally sent periodically.

\subsubsection{PDO protocol}
Process Data Objects (PDOs) are used for broadcasting high-priority status and control information\cite{cia}.
Each PDO consists of a single CAN bus frame and can contain up-to 8 bytes of data.
The contents of the PDO can be set in some devices according to the specific application needs using a technique called PDO mapping where PDO data are mapped to Object Dictionary fields.
There are three mechanisms used to transmit PDOs asynchronous PDOs can be sent upon an event trigger in the device.
Asynchronous PDOs can also be remotely requested using the RTR bit in the CAN frame.
Synchronous PDOs are broadcasted as a reaction to the SYNC protocol.
There are two types of PDOs - RxPDOs and TxPDOs, the RxPDOs are the PDOs that are received by the target device, while the TxPDOs are the PDOs that are transmitted by the target device.
There are four available RxPDOs and four available TxPDOs, each PDO has a CAN ID assigned as can be seen in the Table~\ref{tab:pdo}.

\begin{table}[H]
    \centering
    \begin{tabular}{ |c|c|c| }
        \hline
        PDO & RxPDO CAN ID & TxPDO CAN ID \\
        \hline
        \hline
        PDO1 & 0x200 & 0x180 \\
        \hline
        PDO2 & 0x300 & 0x280 \\
        \hline
        PDO3 & 0x400 & 0x380 \\
        \hline
        PDO4 & 0x500 & 0x480 \\
        \hline
    \end{tabular}
    \caption{RxPDO and TxPDO CAN IDs\cite{wiki_canopen}}
    \label{tab:pdo}
\end{table}

\subsubsection{SDO protocol}

\subsubsection{NMT protocol}

\subsection{I2C}
\label{subsec:i2c}

\subsection{Universal Serial Bus}
\label{subsec:usb}

\section{Electronics Design}
\label{sec:hardware}

This section describes the of the electronics of the SM4 stepper motor controller.
First, the critical components are selected and preliminary design is done based on the requirements stated in the Section~\ref{sec:requirements}.
Second, both of the hardware revisions and their design are described.

In the Figure~\ref{fig:sm4diagram}, we can see the block diagram of the controller.
The central part is the \acs{mcu}, which is connected to the two stepper motor drivers and encoders (hardware encoders are only available in the second electronics revision).
Further there are also peripheral chips and components used to utilize the \acs{can}, I\textsuperscript{2}C and \acs{usb} buses.
The whole controller is powered by the power system providing correct voltages and electrical protection.

\begin{figure}[H]
    \centering
    \includegraphics[width=\textwidth]{obrazky/sm4_block_diagram}
    \caption{The block diagram of the SM4 stepper motor controller.}
    \label{fig:sm4diagram}
\end{figure}


\subsection{Hardware Design Choices}
\label{subsec:hardware_design_choices}
In this section, we describe the choices made in the beginning of the design process, ones that are vital to the functionality of the whole system.
The design choices are based on the requirements stated in the Section~\ref{sec:requirements}, the related work described in the Chapter~\ref{ch:related_work} and our prior experience.

\subsubsection{MCU}
\label{subsubsec:mcu}
The most critical component of the stepper controller is the \acs{mcu}.
The MCU needs to accommodate for the outer communication interfaces as well as the internal ones.
That means that as for the outer communication interfaces, it needs to have peripherals for CAN bus, I\textsuperscript{2}C and USB, as stated in the requirement \textbf{C-03}.
The internal communication interfaces are revision dependent, however the stepper controllers generally require, GPIOs, PWM outputs, serial interfaces and for the possible future encoder support it should require incremental encoder interfaces and SPIs for SSI bitbanging (as stated in requirement \textbf{FR-08}).
As was described in the Chapter on Related work~\ref{ch:related_work}, we decided to make the move from Cortex-M0 and Cortex-M0+ based ARM MCUs to more powerful Cortex-M4 MCUs.
The biggest advantage of these cores is that they fully support atomic instructions, improving memory safety in ISRs, and also that they have FPU (Floating Point Unit).

Given the past experience with STM32 family of ARM microcontrollers, we decided to select the STM32F4 product line, more specifically with the STM32F405RGT6 which features one megabyte of flash and 192 kilobytes of RAM and can be run with the 168 MHz clock\cite{stmicro_stm32f405rg_nodate}.
The block diagram of the MCU with the core features and peripherals can be seen in the Figure~\ref{fig:stm32f405_block_diagram}.

\begin{figure}[H]
    \centering
    \includegraphics[width=0.7\textwidth]{obrazky/stm32f405_block_diagram}
    \caption{The block diagram of the STM32F405RG MCU~\cite{stmicro_enbd_stm32f405_1mbjpg_nodate}.}
    \label{fig:stm32f405_block_diagram}
\end{figure}

This MCU conforms to all the requirement and has enough peripherals to support future development.

\subsection{Stepper motor driver IC}
\label{subsec:stepper_ic}
In order to select a proper driver IC for the SM4 stepper motor controller a simplified comparison was performed.
Since the beginning we were aiming at using the Trinamic made stepper motor driver ICs since there are very good references from the 3D printing community\cite{prusa_original_2017,prusa_original_2019,3daddict_stepper_2020} for them, and we wanted to secure silent function of the motor controller.
With the prior experience with DRV8825 and the A4988 modules described in the Chapter on Related work~\ref{ch:related_work}, we decided to select two drivers from Trinamic - the first hardware revision utilizes the TMC2100-TA as we aimed for as simple chip as possible and also maximum voltage of about 50~V, while the second hardware revision utilizes the TMC2226-SA, which is a more modern, and a fully featured stepper motor driver \acs{ic}, which also has higher peak current.
By selecting these two stepper driver \acs{ic}s, we are conforming to the requirements \textbf{C-01} and \textbf{C-02}.
The comparison can be seen in the following table:

\begin{sidewaystable}
    \centering
    \begin{tabular}{ |p{2.5cm}|p{2cm}|p{2cm}|p{2cm}|p{2.5cm}|p{2.5cm}|p{2cm}|p{4cm}| }
        \hline
        Name & Operating Voltage & Maximum current & Control Interface & Configuration Interface & Microstepping & Package & Advanced Features \\
        \hline
        \hline
        DRV8825\cite{texas_instruments_drv8825_2014} & 8.2-45~V & max 2.5~A (properly cooled, at 24~V, 25~\textdegree) & STEP/DIR & GPIO & up to 32 & HTSSOP28 & None \\
        \hline
        A4988\cite{allegro_microsystems_a4988_2014} & max 35~V & max 2 A & STEP/DIR & GPIO & up to 16 & QFN28ET & None \\
        \hline
        TMC2100\cite{trinamic_tmc2100-datasheet_2018} & max 46~V & max 2 A (2.5 A peaks, properly cooled) & STEP/DIR & GPIO & up to 256 & TQFP48 / QFN36 & MicroPlyer\texttrademark, SpreadCycle\texttrademark, StealthChop\texttrademark \\
        \hline
        TMC2130\cite{trinamic_tmc2130-datasheet_2018} & max 46~V & max 2 A (2.5 A peaks, properly cooled) & STEP/DIR & SPI & up to 256 & TQFP48 / QFN36 & MicroPlyer, SpreadCycle\texttrademark, StealthChop\texttrademark, ChopSync\texttrademark, CoolStep\texttrademark, StallGuard\texttrademark \\
        \hline
        TMC2209\cite{trinamic_tmc2209_2019} & max 29~V & max 2 A (2.8 A peaks) & STEP/DIR & UART & up to 256 & QFN28 & MicroPlyer\texttrademark, SpreadCycle\texttrademark, StealthChop2\texttrademark, CoolStep\texttrademark, StallGuard4\texttrademark \\
        \hline
        TMC2226\cite{trinamic_tmc2226_2020} & max 29~V & max 2 A (2.8 A peaks) & STEP/DIR & UART & up to 256 & HTSSOP28 & MicroPlyer\texttrademark, SpreadCycle, StealthChop2\texttrademark, CoolStep\texttrademark, StallGuard4\texttrademark \\
        \hline
    \end{tabular}
    \caption{Comparison of stepper motor driver ICs}
    \label{tab:driver_ic_comparison}
\end{sidewaystable}
\newpage

% TODO read and refactor
\subsubsection{SM4 power design}
\label{subsubsec:power_design}
The power design of the SM4 stepper motor controller is fairly simple.
According to the requirements, the only requirement for it is to provide basic electrical safety features, such as fuses and reverse voltage protection \textbf{FR-07}.
The controller features two power rails - one for the power electronics, that can utilize quite high voltages and one 5~V for the MCU and the peripheral circuits.
With the first revision, we were considering using a single power-rail with all voltages derived from the power electronics one.
This was however dismissed as a buck converter from quite a high voltage would be required and designing a buck converter is out of the scope of this project and also the motor controller was never meant to be use as a standalone device, meaning that another device could provide the power for the 5~V rail.
The buck converter would also pose EMI (ElectroMagnetic Interference) problems and would increase the price of the motor controller.

As for the power electronics, only input voltage filtering using capacitors was utilized.
The main reasoning being that this should be fused on the side of the power source and that reverse-voltage protection would require quite large components.

The situation is different with the 5~V power rail for peripherals and the MCU.
This power rail utilizes 500~mA PTC fuse, reverse-voltage protection implemented using P-channel MOSFET and a low-pass filter comprising of a ferrite bead and a capacitor.
This power rail is connected to the connectors with CAN bus and I\textsuperscript{2}C.
The output of the filtered power rail is merged with a 5~V power coming from the USB-C connector (which is also fused using a 500~mA PTC fuse) using Schottky diodes.
For powering the MCU with 3.3~V, the 5~V is regulated with an LDO (Low-Dropout) regulator.
The whole power rail can be seen in the schematic in the Figure~\ref{fig:power}.

In the future revisions, the input protection circuits may be replaced by an eFuse\cite{greatscott_best_2021,texas_instruments_efuse_2021}, an IC integrating the input power protection circuits such as overvoltage protection, undervoltage protection, overcurrent protection and reverse-voltage protection.

\begin{figure}[H]
    \centering
    \includegraphics[width=0.9\textwidth]{obrazky/power}
    \caption{The 5~V power rail for powering the MCU and peripherals.}
    \label{fig:power}
\end{figure}

\subsubsection{PCB}
\label{subsubsec:pcb_design}
In order for this project to serve as a testbed for new manufacturing technologies, the \acs{pcb} (\acl{pcb}) was designed as 4-layer.
The ability to design the board as a 4-layer one was enabled by the 4-layer \acs{pcb} manufacturing price decrease by China-based \acs{pcb} manufacturing companies.
Big advantage of designing the \acs{pcb} as 4-layer one was speedup of hardware development - the 4-layer stackup can be utilized so that there is no need to route power to the \acs{ic}s.
In our case we chose the inner layers to be filled with copper planes - one connected to \acs{gnd} and the other one connected to +3.3~V.
This way whenever a connection to +3.3V or \acs{gnd} was required, simply connecting the pad to new via close-by was sufficient.
Apart from being used for power distribution, the large copper planes allow for better \acs{pcb} cooling and also for some minor signal connections in cases routing using the outer layers would prove difficult.
The used stackup can be seen in the Figure~\ref{fig:stackup}.

\begin{figure}[H]
    \centering
    \includegraphics[width=0.9\textwidth]{obrazky/stackup}
    \caption{The 4-layer PCB stackup.}
    \label{fig:stackup}
\end{figure}

Another way to test manufacturing capabilities was utilizing the automated assembly service provided by the China-based PCB manufacturers.
This not-only saved a lot of time with manual assembly, but also enabled us to use smaller components than before - the imperial size 0402.

As for testing out EDA (Electronic Design Aid) software, the KiCAD EDA was used instead of the well-known Eagle.
The KiCAD EDA has improved dramatically in the past years (version 5 and soon to be released version 6), making it great competitor to conventional EDA suites.
The big advantage of KiCAD is a large footprint and symbol library, which often contains even the 3D models and KiCAD itself is able to seamlessly integrate them and render a 3D view of the designed PCB.

\subsection{Schematics and PCB Design using KiCAD}
\label{subsec:kicad}

\subsection{PCB manufacturing using JLCPCB and KiCAD}
\label{subsec:pcb_manu}
\section{Development of the bare-metal firmware}
\label{sec:firmware}
This section describes the development of the bare-metal firmware.
A brief introduction to the Rust programming language is given and the usage of the language for embedded systems is described in detail.
Further, some interesting parts of the firmware itself is described, alongside with the project structure, testing, etc.

\subsection{Rust programming language}
\label{subsec:rust}
Rust is a multi-paradigm systems programming language originally developed by Mozilla\cite{rust_authorship} in an effort to create language suitable for development of a safe and performant multi-threaded CSS rendering engine for the Firefox browser\cite{servo}.
In the recent months the oversight of the language is done by the language's own foundation and is therefore independent on Mozilla\cite{rust_foundation}.

The language itself is designed to be performant and memory efficient - it doesn't feature a garbage collector, memory is managed semi-manually with the leverage of many smart pointer types.
The semi-automatic memory management and its type systems provides guarantees about memory and thread safety, that can be evaluated at compile time, promising that these kinds of potential bugs are found in development rather in production.

The language itself is a part, albeit an important part, of a larger ecosystem, making the language and its tooling extremely usable with tools almost for everything - it features seamless package management and build system, documentation system, integrated testing, defined coding-style and more.

As we said before, the language is a multi-paradigm language, meaning that the language features parts of the functional languages paradigm and object oriented-paradigm.

In the following sections, some features of the language are described in order to provide some introduction into the semantics and syntax of the language.

\subsubsection{Variable and constant expressions}
In Rust, all variables are defined as immutable by default, promoting defensive programming - no variable can be unintentionally changed.
The variables are declared using the keyword \textbf{let} and variable's mutability must be explicitly declared using the \textbf{mut} suffix.
The type of a variably doesn't need to be explicitly specified in most cases as the language features type inference that is possible thanks to its powerful and strong type system.
As for the provided types, the language
An example can be seen in the following code snippet.

\begin{lstlisting}
let a = 10; // declares an immutable variable, whose type is automatically inferred to i32
a = 11; // produces a compile-time error
let mut b: u8 = 0x12; // declares a mutable variable with explicit u8 type
b = 0x24; // this is ok
\end{lstlisting}

Rust also supports compile time constant evaluation using constants and constant functions.
This can be achieved by using the \textbf{const} keyword, but describing this functionality is beyond the scope of this thesis.

\subsection{Embedded Rust}
\label{subsec:embedded_rust}


\subsection{Persistent storage using EEPROM emulation}
\label{subsec:eeprom}
The SM4 stepper motor controller needs persistent storage to save configuration and data.
Persistent storage on MCUs is generally solved by using non-volatile memory that can be either part of the MCU or an external component.
Different memory technologies may be used for both types of the storage.
In general, FRAM (Ferroelectric Random Access Memory), EEPROM (Electrically Erasable programmable read-only memory), or flash memories are used.

In order to save space on the PCB, save cost, and better utilize the MCU resources we decided to use the internal flash memory to store the user data apart from the driver firmware.
Even though the flash memory may seem straightforward to use since they are ubiquitous, their low level use is not that simple.
A flash memory is generally divided into sectors, that can be several kilobytes or megabytes large.
These sectors can be electrically erased - which means that every bit in the sector is set to 1.
Depending on the memory a word of a specific size can be programmed, but it is only possible to flip the bits in the word to zero~\cite{pablo_mansanet_ecorax_nodate}.
That means that to write a higher number to the word of the memory, the flash needs to be first erased and then programmed.
This is problematic for two reasons:
\begin{enumerate}
    \item sectors generally have the size of several kilobytes, meaning that when you'd want to update the value in the desired word, the whole sector would have to be read to some other memory, erased and then programmed again with the new, updated value,
    \item there is a limited number of whole sector erases, caused by the limitation of hardware.
\end{enumerate}

Fortunately, this problem can be solved by emulating the EEPROM memory as described in ST Application Note AN 3969~\cite{stmicro_an3969_2011}.
The application note leverages two flash sectors of the same size, where one of them is marked as the active one and the second one is used when the first sector is full.
The working principle is described in the following paragraphs and can be seen in the Figure~\ref{fig:eeprom_emul}.

In the beginning, both of the sectors are erased and one of them is marked as active.
Data are then written to the first sector into simulated cells.
The cells contain a header (which can be understood as a key or a virtual address) and the data.
When a new write is requested the data are appended behind the already stored data.
When a data with is read using the virtual address or a key, the sector is traversed from its end, searching for the first occurrence of the key or address.
The first occurrence is the most recent value of the cell marked by the key.
This way we are able to store the value with a specific identifier (key, virtual address) in the flash multiple times.

When no more cells can be written to the active sector, the second sector is marked active and the data are transmitted to the second sector, taking only the latest value of an identifier into account.
After the transfer, the first sector is erased.

\begin{figure}[H]
    \centering
    \includegraphics[width=\textwidth]{obrazky/eeprom_emul_principle}
    \caption{EEPROM emulation working principle~\cite{stmicro_an3969_2011}.}
    \label{fig:eeprom_emul}
\end{figure}

Even though the working principle of the EEPROM emulation is simple, there are some technical obstacles in the implementation.
The first obstacle is that the flash memory on the STM32 MCU is split into differently sized sectors and it is required that the sectors have the same size.
Referring to the Reference Manual~\cite{stmicro_stm32f405rg_nodate} there are a some 16~kilobyte sectors that could be used for the emulation, as can be seen in the Figure~\ref{fig:flash_layout}.
Using the 128~kilobyte sectors would also be possible, but given their size copying values from one sector to another would take too much time and also read access times would be higher.

\begin{figure}[H]
    \centering
    \includegraphics[width=0.8\textwidth]{obrazky/flash_stm}
    \caption{EEPROM emulation working principle~\cite{stmicro_stm32f405rg_nodate}.}
    \label{fig:flash_layout}
\end{figure}

There is however a problem with using the sectors in the beginning of the flash memory as that is where the firmware is usually stored.
The solution to this problem is by leaving the first sector (Sector 0) for the vector table and instructing the linker to place the \textbf{.text} section of the program further in the memory.
According to the documentation of the \textbf{cortex-m} Rust crate~\cite{rust_embedded_wg_rust-embeddedcortex-m-rt_nodate}, this can be achieved by adding the line \textbf{\_stext = ORIGIN(FLASH) + OFFSET} to the linker script, where the \textbf{OFFSET} shall be replaced with the offset of the target sector where we want our program to be stored, in our case \textbf{0x0000C000}, which indicates the start of the Sector 3.

As for the actual implementation of the emulation for the STM32F405, we decided to develop our own, as no suitable Rust crate was available for it.
The development was inspired by a crate that implemented the emulation for STM32F103~\cite{crate_f103_eeprom} and by following the implementation in the Application Note.
The functions from flash memory access were adopted from an as of the time of writing unmerged Pull Request into the STM32F4 HAL~\cite{astro_implement_2020}.
An example of accessing the emulated persistent storage can be seen in the following Listing~\ref{lst:eeprom}.

\begin{lstlisting}[caption={An example use of the emulated persistent storage.},label=lst:eeprom]
let mut store = Storage::new(device.FLASH);
store
    .init()
    .expect("Failed to initialize emulated storage.");
store.write_f32(0xbeef, 3.14);
let read = store.read_f32(0xbeef).unwrap();
assert_eq!(read, 3.14);
\end{lstlisting}

As can be seen in the Listing~\ref{lst:eeprom}, first we create the object with a parameter of the flash peripheral, then we initialize the emulated storage - this prepares the sectors that are supposed to be used and then we perform a simple write and read operations on the storage.

\section{Development of the control application}


\chapter{Results}
\label{ch:results}
This chapter discusses the implemented functionality, features and the final state of the project in general.
Apart from the final project state, two demonstrations are showcased - one of them being a small mobile robot for indoor mapping, and the second one being a stepper motor driven linear rail useful for example for camera movement.

\section{Final Project State}
\label{sec:final_project_state}

\subsection{Takeaways for Future Revisions}
\label{subsec:final_takeaways}
In this section, we describe what changes we'd made in a possible future revisions.
The changes are:
\begin{itemize}
    \item Use XT-30 connector for motor power.
    \item Use different connectors for motors.
    \item Use standardized 10 pin JTAG connector for SWD.
    \item Use eFuse for electronics protection.
    \item Add more status LEDs.
    \item Improve encoder connector placement, select appropriate connectors.
    \item Remove compatibility resistors around the CAN transceiver.
    \item Attempt to utilize async Rust for easier development.
\end{itemize}

\section{Demonstration \#1 - Small Mobile Robot for Indoor Mapping}
\label{sec:dem1}
\epigraph{
    Any exploration program which "just happens" to include a new launch vehicle is, de facto, a launch vehicle program. \\ \\
    (alternate formulation) The three keys to keeping a new human space program affordable and on schedule: \\
1)  No new launch vehicles. \\
2)  No new launch vehicles. \\
3)  Whatever you do, don't develop any new launch vehicles.}{Akin's Laws of Spacecraft Design\cite{david_l_akin_akins_nodate}}

The first demonstration, where we showcase the SM4 stepper motor controller, is a small differentially driven robot aimed for indoor mapping and self-localization.
The chassis is differential with one motor on two sides of the robot, which showcases the driver's ability to control both of the motor and calculate odometry.
Apart from the driver itself, the robot features a Raspberry Pi 4B SBC, 4 cell Li-Ion battery, step-down converter and a simple planar LIDAR for the mapping task.
It is projected, that the robot will be utilized for algorithm demonstration as part of the MPC-MAP - Advanced Mapping and Self-Localization for Robotics course.
The finalized robot can be seen in the Figure~\ref{fig:map_bot}.

\begin{figure}[H]
    \centering
    \includegraphics[width=0.7\textwidth]{obrazky/map_bot}
    \caption{The MAP-bot robot - the first demonstration of the SM4 stepper motor controller.}
    \label{fig:map_bot}
\end{figure}

As part of the demonstration, the Raspberry Pi SBC controls the robot to move forward, backward and rotate via CANOpen control of the SM4 stepper motor controller.

\section{Demonstration \#2 - Linear Rail Actuator for Camera Movement}
\label{sec:dem2}

\chapter*{Conclusion, Discussion and Future work}
\phantomsection
\addcontentsline{toc}{chapter}{Conclusion, Discussion and Future work}

In this thesis, we described the development of a dual channel stepper motor controller we named SM4.
We developed both the hardware and software.
As for the hardware development we utilized the STM32F405 MCU, Trinamic stepper motor driver ICs as the basis of the design.
As for PCB design, we utilized a 4-layer PCB to decrease the development time and we utilized the JLCPCB's manufacturing service alongside with the PCB assembly service.
Two revisions of the PCB were designed and manufactured, two of each revision PCBs were assembled.
The schematic and PCBs were designed in the KiCAD EDA suite, which proved useful, given the KiCAD has large footprint and symbol library.

As for development of the software, we utilized the Rust programming language for both the bare-metal firmware and the control application.
Developing the firmware in Rust proved useful, as there is a great support for developing on bare-metal, given there is a large community for developing device drivers, HALs and tooling.
We believe that the tooling that currently exists surpasses the tooling available for other languages and ecosystems.
The language's features provide memory and data race safety for embedded systems while not impeding the code size or performance.
Given our experience with developing embedded firmware for this project and several other ones described earlier, we firmly believe that the Rust programming languages is the right way forward as it brings features never deemed possible for embedded systems development.

Apart from the firmware, we also developed a simple control application for personal computers, that is now capable of only controlling the stepper motor controller but not of configuring it.

As for the future work, there is a lot to do.
First, some of the requirements were not completely fulfilled, unfortunately some of them are crucial.
Also, the documentation and tests are definitely not completed which creates an obstacle for future development.
An important thing is that the firmware as of now doesn't have support for the stepper motor controller ICs used in the second revision of the hardware.
This is a problem, given mostly by the fact that these controller ICs are much more advanced than the ones used in the first revision.
In the future, we would also like to fully utilize the capabilities of the second hardware revision and use the encoder interface to precisely measure the shaft position.
When the stepper motor controller will be used by the target developers, we expect to get vital feedback about the usability of the provided APIs.
We will also aim for full conformance with the CANOpen standard, to allow for seamless integration with other systems.
We are also looking into continuing the development using Rust programming language for embedded systems, either by developing more projects using it or contributing to the existing ecosystem by developing the tooling and libraries.


\bibliographystyle{unsrturl}
\bibliography{text/literatura}
\nocite{*}

\cleardoublepage
\chapter*{\listofabbrevname}
\phantomsection
\addcontentsline{toc}{chapter}{\listofabbrevname}

\begin{acronym}[KolikMista]

	\acro{adc}[ADC]{Analog to Digital Converter}
	\acro{ai}[AI]{Artificial Intelligence}
	\acro{api}[API]{Application Programming Interface}
	\acro{arm}[ARM]{Acorn RISC Machine}
	\acro{can}[CAN]{Controlled Area Network}
	\acro{ci}[CI]{Continuous Integration}
	\acro{dc}[DC]{Direct Current}
	\acro{dma}[DMA]{Direct Memory Access}
	\acro{dsl}[DSL]{Domain Specific Language}
	\acro{dfu}[DFU]{Device Firmware Update}
	\acro{emi}[EMI]{ElectroMagnetic Interference}
	\acro{esd}[ESD]{ElectroStatic Discharge}
	\acro{ffi}[FFI]{Foreign Function Interface}
	\acro{gnd}[GND]{Ground}
	\acro{gpio}[GPIO]{General Purpose Input Output}
	\acro{hal}[HAL]{Hardware Abstraction Layer}
	\acro{ic}[IC]{Integrated Circuit}
	\acro{ide}[IDE]{Integrated Development Environment}
	\acro{led}[LED]{Light Emitting Diode}
	\acro{lidar}[LIDAR]{Light Detection and Ranging}
	\acro{mcu}[MCU]{Microcontroller Unit}
	\acro{no}[NO]{Normally Open}
	\acro{pac}[PAC]{Peripheral Access Crate}
	\acro{pcb}[PCB]{Printed Circuit Board}
	\acro{ptc}[PTC]{Positive Temperature Coefficient}
	\acro{qa}[QA]{Quality Assurance}
	\acro{ram}[RAM]{Random Access Memory}
	\acro{ral}[RAL]{Register Access Layer}
	\acro{risc}[RISC]{Reduced Instruction Set Computer}
	\acro{rms}[RMS]{Root-Mean Square}
	\acro{rtic}[RTIC]{Real-Time Interrupt Driven Concurrency}
	\acro{rtt}[RTT]{Real-Time Transfer}
	\acro{sbc}[SBC]{Single Board Computer}
	\acro{spi}[SPI]{Serial Peripheral Interface}
	\acro{spdt}[SPDT]{Single Pole Double Throw}
	\acro{svd}[SVD]{System View Description}
	\acro{ub}[UB]{Undefined Behavior}
	\acro{usb}[USB]{Universal Serial Bus}


\end{acronym}


\appendix
%%% Vysázení seznamu příloh
% (vynechejte, pokud máte dvě nebo méně příloh)
\listofappendices

%%% Vložení souboru 'text/prilohy' s přílohami
% Obvykle je přítomen alespoň popis co najdeme na přiloženém médiu
%\chapter{Některé příkazy balíčku \texttt{thesis}}
%
%\section{Příkazy pro sazbu veličin a jednotek}
%
%\begin{table}[!h]
%  \caption[Přehled příkazů]{Přehled příkazů pro matematické prostředí }
%  \begin{center}
%  	\small
%	  \begin{tabular}{|c|c|c|c|}
%	    \hline
%	    Příkaz    						& Příklad 					& Zdroj příkladu  							& Význam  \\
%	    \hline\hline
%	    \verb|\textind{...}|	& $\beta_\textind{max}$ 	& \verb|$\beta_\textind{max}$|	& textový index \\
%	    \hline
%	    \verb|\const{...}| 		& $\const{U}_\textind{in}$ 				& \verb|$\const{U}_\textind{in}$|		& konstantní veličina \\
%	    \hline
%	    \verb|\var{...}| 		& $\var{u}_\textind{in}$ & \verb|$\var{u}_\textind{in}$| & proměnná veličina \\
%	    \hline
%	    \verb|\complex{...}| 	& $\complex{u}_\textind{in}$ & \verb|$\complex{u}_\textind{in}$| & komplexní veličina \\
%	    \hline
%	    \verb|\vect{...}| 		& $\vect{y}$ 						& \verb|$\vect{y}$| & vektor \\
%	    \hline
%	    \verb|\mat{...}| 	& $\mat{Z}$ 						& \verb|$\mat{Z}$| & matice \\
%	    \hline
%	    \verb|\unit{...}| 		& $\unit{kV}$ 						& \verb|$\unit{kV}$|\quad či\ \, \verb|\unit{kV}| & jednotka \\
%	    \hline
%	  \end{tabular}
%  \end{center}
%\end{table}
%
%
%
%%\newpage
%\section{Příkazy pro sazbu symbolů}
%
%\begin{itemize}
%  \item
%    \verb|\E|, \verb|\eul| -- sazba Eulerova čísla: $\eul$,
%  \item
%    \verb|\J|, \verb|\jmag|, \verb|\I|, \verb|\imag| -- sazba imaginární jednotky: $\jmag$, $\imag$,
%  \item
%    \verb|\dif| -- sazba diferenciálu: $\dif$,
%  \item
%    \verb|\sinc| -- sazba funkce: $\sinc$,
%  \item
%    \verb|\mikro| -- sazba symbolu mikro stojatým písmem%
%			\footnote{znak pochází z~balíčku \texttt{textcomp}}: $\mikro$,
%	\item
%		\verb|\uppi| -- sazba symbolu $\uppi$
%			(stojaté řecké pí, na rozdíl od \verb|\pi|, což sází $\pi$).
%\end{itemize}
%%
%Všechny symboly jsou určeny pro matematický mód, vyjma \verb|\mikro|, jenž je\\ použitelný rovněž v~textovém módu.
%%$\upmikro$
%
%
%\chapter{Druhá příloha}
%
%\begin{figure}[!h]
%  \begin{center}
%    \includegraphics[scale=0.5]{obrazky/ZlepseneWilsonovoZrcadloNPN}
%  \end{center}
%  \caption[Alenčino zrcadlo]{Zlepšené Wilsonovo proudové zrcadlo.}
%\end{figure}
%
%Pro sazbu vektorových obrázků přímo v~\LaTeX{}u je možné doporučit balíček \href{https://www.ctan.org/pkg/pgf}{\texttt{TikZ}}.
%Příklady sazby je možné najít na \href{http://www.texample.net/tikz/examples/}{\TeX{}ample}.
%Pro vyzkoušení je možné použít programy QTikz nebo TikzEdt.
%
%
%
%
%\chapter{Příklad sazby zdrojových kódů}
%
%\section{Balíček \texttt{listings}}
%
%Pro vysázení zdrojových souborů je možné použít balíček \href{https://www.ctan.org/pkg/listings}{\texttt{listings}}.
%Balíček zavádí nové prostředí \texttt{lstlisting} pro sazbu zdrojových kódů, jako například:
%%
%\begin{lstlisting}[language={[LaTeX]TeX}]
%\section{Balíček lstlistings}
%Pro vysázení zdrojových souborů je možné použít
%	balíček \href{https://www.ctan.org/pkg/listings}%
%	{\texttt{listings}}.
%Balíček zavádí nové prostředí \texttt{lstlisting} pro
%	sazbu zdrojových kódů.
%\end{lstlisting}
%%
%Podporuje množství programovacích jazyků.
%Kód k~vysázení může být načítán přímo ze zdrojových souborů.
%Umožňuje vkládat čísla řádků nebo vypisovat jen vybrané úseky kódu.
%Např.:
%
%\noindent
%Zkratky jsou sázeny v~prostředí \texttt{acronym}:
%\label{lst:zkratky}
%\lstinputlisting[language={[LaTeX]TeX},nolol,numbers=left, firstnumber=6, firstline=6,lastline=6]{text/zkratky.tex}
%%
%Šířka textu volitelného parametru \verb|KolikMista| udává šířku prvního sloupce se zkratkami.
%Proto by měla být zadávána nejdelší zkratka nebo symbol.
%Příklad definice zkratky \acs{symfvz} je na výpisu \ref{lst:symfvz}.
%
%\shorthandoff{-}
%\lstinputlisting[language={[LaTeX]TeX},frame=single,caption={Ukázka sazby zkratek},label=lst:symfvz,numbers=left,linerange={bsymfvz-\%\%\%\ esymfvz},includerangemarker=false]{text/zkratky.tex}
%\shorthandon{-}
%
%\noindent
%Ukončení seznamu je provedeno ukončením prostředí:
%\lstinputlisting[language={[LaTeX]TeX},nolol,numbers=left,firstnumber=26,linerange=26]{text/zkratky.tex}
%
%\vspace{\fill}
%
%\noindent
%{\bf Poznámka k~výpisům s~použitím volby jazyka \verb|czech| nebo \verb|slovak|:}\newline
%Pokud Váš zdrojový kód obsahuje znak spojovníku \verb|-|, pak překlad může skončit chybou.
%Ta je způsobená tím, že znak \verb|-| je v~českém nebo slovenském nastavení balíčku \verb|babel| tzv.\ aktivním znakem.
%Přepněte znak \verb|-| na neaktivní příkazem \verb|\shorthandoff{-}| těsně před výpisem a hned za ním jej vraťte na aktivní příkazem \verb|\shorthandon{-}|.
%Podobně jako to je ukázáno ve zdrojovém kódu šablony.
%
%
%\clearpage
%
%%\section{Výpis kódu prostředí Matlab}
%Na výpisu \ref{lst:priklad.vypis.kodu.Matlab} naleznete příklad kódu pro Matlab, na výpisu \ref{lst:priklad.vypis.kodu.C} zase pro jazyk~C.
%
%\lstnewenvironment{matlab}[1][]{%
%\iflanguage{czech}{\shorthandoff{-}}{}%
%\iflanguage{slovak}{\shorthandoff{-}}{}%
%\lstset{language=Matlab,numbers=left,#1}%
%}{%
%\iflanguage{slovak}{\shorthandon{-}}{}%
%\iflanguage{czech}{\shorthandon{-}}{}%
%}
%
%\begin{matlab}[frame=single,float=htbp,caption={Příklad Schur-Cohnova testu stability v~prostředí Matlab.},label=lst:priklad.vypis.kodu.Matlab,numberstyle=\scriptsize, numbersep=7pt]
%%% Priklad testovani stability filtru
%
%% koeficienty polynomu ve jmenovateli
%a = [ 5, 11.2, 5.44, -0.384, -2.3552, -1.2288];
%disp( 'Polynom:'); disp(poly2str( a, 'z'))
%
%disp('Kontrola pomoci korenu polynomu:');
%zx = roots( a);
%if( all( abs( zx) < 1))
%    disp('System je stabilni')
%else
%    disp('System je nestabilni nebo na mezi stability');
%end
%
%disp(' '); disp('Kontrola pomoci Schur-Cohn:');
%ma = zeros( length(a)-1,length(a));
%ma(1,:) = a/a(1);
%for( k = 1:length(a)-2)
%    aa = ma(k,1:end-k+1);
%    bb = fliplr( aa);
%    ma(k+1,1:end-k+1) = (aa-aa(end)*bb)/(1-aa(end)^2);
%end
%
%if( all( abs( diag( ma.'))))
%    disp('System je stabilni')
%else
%    disp('System je nestabilni nebo na mezi stability');
%end
%\end{matlab}
%
%\noindent
%\begin{minipage}{\linewidth}
%
%
%%\section{Výpis kódu jazyka C}
%
%\begin{lstlisting}[frame=single,numbers=right,caption={Příklad implementace první kanonické formy v~jazyce C.},label=lst:priklad.vypis.kodu.C,basicstyle=\ttfamily\small, keywordstyle=\color{black}\bfseries\underbar,]
%// první kanonická forma
%short fxdf2t( short coef[][5], short sample)
%{
%	static int v1[SECTIONS] = {0,0},v2[SECTIONS] = {0,0};
%	int x, y, accu;
%	short k;
%
%	x = sample;
%	for( k = 0; k < SECTIONS; k++){
%		accu = v1[k] >> 1;
%		y = _sadd( accu, _smpy( coef[k][0], x));
%		y = _sshl(y, 1) >> 16;
%
%		accu = v2[k] >> 1;
%		accu = _sadd( accu, _smpy( coef[k][1], x));
%		accu = _sadd( accu, _smpy( coef[k][2], y));
%		v1[k] = _sshl( accu, 1);
%
%		accu = _smpy( coef[k][3], x);
%		accu = _sadd( accu, _smpy( coef[k][4], y));
%		v2[k] = _sshl( accu, 1);
%
%		x = y;
%	}
%	return( y);
%}
%\end{lstlisting}
%\end{minipage}
%
%
%
%
%
%
%
%\chapter{Obsah elektronické přílohy}
%Elektronická příloha je často nedílnou součástí semestrální nebo závěrečné práce.
%Vkládá se do informačního systému VUT v~Brně ve vhodném formátu (ZIP, PDF\,\dots).
%
%Nezapomeňte uvést, co čtenář v~této příloze najde.
%Je vhodné okomentovat obsah každého adresáře, specifikovat, který soubor obsahuje důležitá nastavení, který soubor je určen ke spuštění, uvést nastavení kompilátoru atd.
%Také je dobře napsat, v~jaké verzi software byl kód testován (např.\ Matlab 2018b).
%Pokud bylo cílem práce vytvořit hardwarové zařízení,
%musí elektronická příloha obsahovat veškeré podklady pro výrobu (např.\ soubory s~návrhem DPS v~Eagle).
%
%Pokud je souborů hodně a jsou organizovány ve více složkách, je možné pro výpis adresářové struktury použít balíček \href{https://www.ctan.org/pkg/dirtree}{\texttt{dirtree}}.
%
%\bigskip
%
\chapter{Contents of the Enclosed Electronic Appendix}
\label{ch:flash_contents}
{\small
%
\dirtree{%.
.1 /\DTcomment{Root}.
.2 .github\DTcomment{CI workflows and auxiliary files}.
.2 Hardware\DTcomment{Hardware resources for both HW revisions}.
.3 Cube\DTcomment{STM32CubeMX project for pin assigment}.
.3 Docs\DTcomment{Datasheets of components}.
.3 Libs\DTcomment{KiCAD component and footprint libraries}.
.3 rev1\DTcomment{KiCAD project for the first revision}.
.3 rev2\DTcomment{KiCAD project for the second revision}.
.2 Poster\DTcomment{Source file for a future poster}.
.3 poster\_template.pptx.
.2 Software\DTcomment{Software projects and source code}.
.3 controller\DTcomment{The control software}.
.3 embedded\DTcomment{Workspace containing cross-compiled projects}.
.4 firmware\DTcomment{The motor controller's firmware}.
.4 testsuite\DTcomment{Tests for the motor controller's firmware}.
.3 shared\DTcomment{Project with code shared between firmware and controller}.
.3 Cargo.toml\DTcomment{Cargo project file}.
.3 Cargo.lock\DTcomment{Cargo project file lock}.
.3 LICENSE-MIT\DTcomment{Software License}.
.3 README.md\DTcomment{Read me for software}.
.2 Thesis\DTcomment{Source code for this thesis}.
.2 README.md\DTcomment{Readme for this master's project}.
}
}

\chapter{Schematic of the Second Electronics Revision}
\label{ch:rev2_schematic}
\includepdf[pages=1, landscape=true]{pdf/schematic}

\chapter{PCB of the Second Electronics Revision}
\label{ch:rev2_pcb}
\includepdf[pages=-, landscape=true]{pdf/pcb}

\chapter{CANOpen PDOs and Object Dictionary}
\label{ch:canopen_appendices}
\begin{table}[H]
    \centering
    \begin{tabular}{ |p{1.5cm}|p{1.8cm}|p{1.8cm}|p{2cm}|p{5.5cm}| }
        \hline
        Index & Subindex & Type & Length [B] & Description \\
        \hline
        \hline
        2000 & 1 & f32 & 4 & Battery Voltage in Volts \\
        \hline
        2000 & 2 & f32 & 4 & MCU Temperature in \textdegree C \\
        \hline
        2[1,2]00 & 1 & AxisMode & 1 & mode of the axis - 0 for velocity, 1 for position \\
        \hline
        2[1,2]00 & 2 & bool & 1 & axis enabled \\
        \hline
        2[1,2]00 & 3 & f32 & 4 & target axis velocity in RPS \\
        \hline
        2[1,2]00 & 4 & f32 & 4 & actual axis velocity in RPS \\
        \hline
        2[1,2]00 & 5 & i32 & 4 & target axis position - revolutions \\
        \hline
        2[1,2]00 & 6 & u32 & 4 & target axis position - angle \\
        \hline
        2[1,2]00 & 7 & i32 & 4 & actual axis position - revolutions \\
        \hline
        2[1,2]00 & 8 & u32 & 4 & actual axis position - angle \\
        \hline
        2[1,2]00 & 9 & f32 & 4 & target ramp acceleration in RPS per second \\
        \hline
        2[1,2]00 & 10 & bool & 1 & velocity controller bypass enabled \\
        \hline
        2[1,2]00 & 11 & f32 & 4 & current applied to motor winding during acceleration in Amperes \\
        \hline
        2[1,2]00 & 12 & f32 & 4 & current applied to motor winding when idle in Amperes \\
        \hline
        2[1,2]00 & 13 & f32 & 4 & current applied to motor winding when moving with constant speed in Amperes \\
        \hline
        2[1,2]00 & 14 & f32 & 4 & velocity controller proportional gain \\
        \hline
        2[1,2]00 & 15 & f32 & 4 & velocity controller summation gain \\
        \hline
        2[1,2]00 & 16 & f32 & 4 & velocity controller differential gain \\
        \hline
        2[1,2]00 & 17 & f32 & 4 & velocity controller maximal action value \\
        \hline
        2[1,2]00 & 18 & f32 & 4 & position controller proportional gain \\
        \hline
        2[1,2]00 & 19 & f32 & 4 & position controller summation gain \\
        \hline
        2[1,2]00 & 20 & f32 & 4 & position controller differential gain \\
        \hline
        2[1,2]00 & 21 & f32 & 4 & position controller maximal action value \\
        \hline
    \end{tabular}
    \caption{RxPDO1 mapping}
    \label{tab:object_dictionary}
\end{table}

\begin{table}[H]
    \centering
    \begin{tabular}{ |p{3cm}|p{2cm}|p{8cm}| }
        \hline
        Value & Length [B] & Description \\
        \hline
        \hline
        axis mode & 1 & LSB contains axis 1 mode - 0 means velocity mode, 1 means position mode, first bit of the second nimble contains axis 2 mode \\
        \hline
        axis enabled & 1 & LSB sets axis 1 enabled - 0 means disabled, 1 means enabled, second lowest bit sets axis 2 enabled \\
        \hline
    \end{tabular}
    \caption{RxPDO1 mapping}
    \label{tab:rxpdo1}
\end{table}

\begin{table}[H]
    \centering
    \begin{tabular}{ |p{3cm}|p{2cm}|p{8cm}| }
        \hline
        Value & Length [B] & Description \\
        \hline
        \hline
        battery voltage & 2 & battery voltage in millivolts \\
        \hline
        temperature & 2 & temperature in tenths of \textdegree C \\
        \hline
    \end{tabular}
    \caption{TxPDO1 mapping}
    \label{tab:txpdo1}
\end{table}


RxPDO2, TxPDO2
\begin{table}[H]
    \centering
    \begin{tabular}{ |p{3cm}|p{2cm}|p{8cm}| }
        \hline
        Value & Length [B] & Description \\
        \hline
        \hline
        axis 1 velocity & 4 & 32-bit float indicating axis 1 velocity in revolutions per second \\
        \hline
        axis 2 velocity & 4 & 32-bit float indicating axis 2 velocity in revolutions per second \\
        \hline
    \end{tabular}
    \caption{Mapping of PDOs containing velocity information - RxPDO2 and TxPDO2}
    \label{tab:velocity_pdo}
\end{table}

\begin{table}[H]
    \centering
    \begin{tabular}{ |p{3cm}|p{2cm}|p{8cm}| }
        \hline
        Value & Length [B] & Description \\
        \hline
        \hline
        axis revolutions & 4 & signed 32-bit integer denoting the number of axis shaft revolutions \\
        \hline
        axis angle & 4 & unsigned 32-bit indicating axis shaft angle \\
        \hline
    \end{tabular}
    \caption{Mapping of PDOs containing position information - RxPDO3, RxPDO4, TxPDO3 and TxPDO4}
    \label{tab:position_pdo}
\end{table}



\end{document}